\documentclass[11pt]{article}
\usepackage{amssymb}
\usepackage{amsthm}
\usepackage{enumitem}
\usepackage{amsmath}
\usepackage{bm}
\usepackage{adjustbox}
\usepackage{mathrsfs}
\usepackage{graphicx}
\usepackage{siunitx}
\usepackage[mathscr]{euscript}

\title{\textbf{Solved selected problems of Special Relativity - Morin}}
\author{Franco Zacco}
\date{}

\addtolength{\topmargin}{-3cm}
\addtolength{\textheight}{3cm}

\newcommand{\hatr}{\bm{\hat{r}}}
\newcommand{\hatx}{\bm{\hat{x}}}
\newcommand{\haty}{\bm{\hat{y}}}
\newcommand{\hatz}{\bm{\hat{z}}}
\newcommand{\hatth}{\bm{\hat{\theta}}}
\newcommand{\hatphi}{\bm{\hat{\phi}}}
\newcommand{\hatrho}{\bm{\hat{\rho}}}
\theoremstyle{definition}
\newtheorem*{solution*}{Solution}

\begin{document}
\maketitle
\thispagestyle{empty}

\section*{Chapter 2 - Kinematics, Part 2}

	\begin{proof}{\textbf{2.1}}
        By checking the Excercise 1 of Section 1.5 we have that the velocity $v$ and
        the factor $\gamma$ for the different frame combinations are
        \begin{table}[h]
            \centering
            \begin{tabular}{lllllll}
            \hline
                     & AB      & AC     & AD            & BC     & BD            & CD            \\ \hline
            $v$      & $5c/13$ & $4c/5$ & $c/5$         & $3c/5$ & $c/5$         & $5c/7$        \\
            $\gamma$ & $13/12$ & $5/3$  & $5/2\sqrt{6}$ & $5/4$  & $5/2\sqrt{6}$ & $7/2\sqrt{6}$ \\
            \end{tabular}
        \end{table}

        Now let us verify the values of the table we have using the Lorentz
        Transformations
        \begin{align*}
            \Delta x &= \gamma (\Delta x' + v\Delta t')\\
            \Delta t &= \gamma (\Delta t' + v\Delta x'/c^2)
        \end{align*}
        From frame $A$ with $\Delta t_A$ and $\Delta x_A$ we calculate $\Delta t_B$ and
        $\Delta x_B$ for frame $B$ as follows
        \begin{align*}
            \Delta x_B &= \gamma_{AB}(\Delta x_A + v_{AB}\Delta t_A)\\
                &= \frac{13}{12}(-L + \frac{5c}{13}\frac{5L}{c}) \\
                &= \frac{13}{12}(\frac{12}{13}L) = L \\
            \Delta t_B &= \gamma_{AB} (\Delta t_A + v_{AB}\Delta x_A/c^2)\\
                &= \frac{13}{12}(\frac{5L}{c} - \frac{5c}{13}\frac{L}{c^2}) \\
                &= \frac{13}{12}(\frac{60}{13}\frac{L}{c}) = \frac{5L}{c}
        \end{align*}
\cleardoublepage
        From frame $A$ with $\Delta t_A$ and $\Delta x_A$ we calculate $\Delta t_C$ and
        $\Delta x_C$ for frame $C$ as follows
        \begin{align*}
            \Delta x_C &= \gamma_{AC}(\Delta x_A + v_{AC}\Delta t_A)\\
                &= \frac{5}{3}(-L + \frac{4c}{5}\frac{5L}{c}) \\
                &= \frac{5}{3}(3L) = 5L \\
            \Delta t_C &= \gamma_{AC} (\Delta t_A + v_{AC}\Delta x_A/c^2)\\
                &= \frac{5}{3}(\frac{5L}{c} - \frac{4c}{5}\frac{L}{c^2}) \\
                &= \frac{5}{3}(\frac{21}{3}\frac{L}{c}) = \frac{7L}{c}
        \end{align*}
        From frame $A$ with $\Delta t_A$ and $\Delta x_A$ we calculate $\Delta t_D$ and
        $\Delta x_D$ for frame $D$ as follows
        \begin{align*}
            \Delta x_D &= \gamma_{AD}(\Delta x_A + v_{AD}\Delta t_A)\\
                &= \frac{5}{2\sqrt{6}}(-L + \frac{c}{5}\frac{5L}{c}) \\
                &= \frac{5}{2\sqrt{6}} \cdot 0 = 0 \\
            \Delta t_D &= \gamma_{AD} (\Delta t_A + v_{AD}\Delta x_A/c^2)\\
                &= \frac{5}{2\sqrt{6}}(\frac{5L}{c} - \frac{c}{5}\frac{L}{c^2}) \\
                &= \frac{5}{2\sqrt{6}}(\frac{24}{5}\frac{L}{c}) = \frac{2\sqrt{6}L}{c}
        \end{align*}
        From frame $B$ with $\Delta t_B$ and $\Delta x_B$ we calculate $\Delta t_C$ and
        $\Delta x_C$ for frame $C$ as follows
        \begin{align*}
            \Delta x_C &= \gamma_{BC}(\Delta x_B + v_{BC}\Delta t_B)\\
                &= \frac{5}{4}(L + \frac{3c}{5}\frac{5L}{c}) \\
                &= \frac{5}{4}(4L) = 5L \\
            \Delta t_C &= \gamma_{BC} (\Delta t_B + v_{BC}\Delta x_B/c^2)\\
                &= \frac{5}{4}(\frac{5L}{c} + \frac{3c}{5}\frac{L}{c^2}) \\
                &= \frac{5}{4}(\frac{28}{5}\frac{L}{c}) = \frac{7L}{c}
        \end{align*}
\cleardoublepage
        From frame $B$ with $\Delta t_B$ and $\Delta x_B$ we calculate $\Delta t_D$ and
        $\Delta x_D$ for frame $D$ taking into account that in this case $B$ moves
        to the left so the sign of the second term is negative then 
        \begin{align*}
            \Delta x_D &= \gamma_{BD}(\Delta x_B - v_{BD}\Delta t_B)\\
                &= \frac{5}{2\sqrt{6}}(L - \frac{c}{5}\frac{5L}{c}) \\
                &= \frac{5}{2\sqrt{6}} \cdot 0 = 0 \\
            \Delta t_D &= \gamma_{BD} (\Delta t_B - v_{BD}\Delta x_B/c^2)\\
                &= \frac{5}{2\sqrt{6}}(\frac{5L}{c} - \frac{c}{5}\frac{L}{c^2}) \\
                &= \frac{5}{2\sqrt{6}}(\frac{24}{5}\frac{L}{c}) = \frac{2\sqrt{6}L}{c}
        \end{align*}
        From frame $C$ with $\Delta t_C$ and $\Delta x_C$ we calculate $\Delta t_D$ and
        $\Delta x_D$ for frame $D$ taking into account that in this case $C$ moves
        to the left so the sign of the second term is negative then 
        \begin{align*}
            \Delta x_D &= \gamma_{CD}(\Delta x_C - v_{CD}\Delta t_C)\\
                &= \frac{7}{2\sqrt{6}}(5L - \frac{5c}{7}\frac{7L}{c}) \\
                &= \frac{7}{2\sqrt{6}} \cdot 0 = 0 \\
            \Delta t_D &= \gamma_{CD} (\Delta t_C - v_{CD}\Delta x_C/c^2)\\
                &= \frac{7}{2\sqrt{6}}(\frac{7L}{c} - \frac{5c}{7}\frac{5L}{c^2}) \\
                &= \frac{7}{2\sqrt{6}}(\frac{24}{7}\frac{L}{c}) = \frac{2\sqrt{6}L}{c}
        \end{align*}
    \end{proof}

\cleardoublepage
    \begin{proof}{\textbf{2.2}}
        Since the photon is moving vertically and you are moving to the left with a speed
        $v$ with respect to the ground then from your frame and using the velocity
        addition formula the photon is moving with the following velocities
        \begin{align*}
            u_x = \frac{0 + v}{1 + 0} = v \quad\quad\quad\quad
            u_y = \frac{c}{\gamma_v(1 + 0)} = \frac{c}{\gamma_{v}}
        \end{align*}
        If we want that the photon travels $45^{\circ}$ rightward and upward from your
        frame then the velocities $u_x$ and $u_y$ must be equal, then
        \begin{align*}
            v &= \frac{c}{\gamma_v}\\
            \frac{v^2}{c^2} &= 1 - \frac{v^2}{c^2}\\
            v^2 &= \frac{c^2}{2}
        \end{align*}
        Therefore $v = \frac{1}{\sqrt{2}}c$
    \end{proof}
	\begin{proof}{\textbf{2.3}}
        Let us have a frame $S'$ which is travelling along the dotted line with the speed
        $v \cos\theta$ measured from the lab frame ($S$) in the particle's opposite
        $x$ direction. Then to change from $S'$ to $S$ we need to use the following Lorentz
        factor
        $$\gamma = \frac{1}{\sqrt{1 - \frac{v^2}{c^2}\cos^2\theta}}$$
        Now, seen from the frame $S'$ the particles are actually travelling upward and
        downward. So seen from the lab's frame ($S$) we have that
        $$v\sin\theta = u_y = \frac{u_y'}{\gamma(1 + 0)} = \frac{u_y'}{\gamma}$$
        Therefore in $S'$ each particle moves with velocity
        $$u_y' = \gamma v\sin\theta = \frac{v\sin\theta}{\sqrt{1 - \frac{v^2}{c^2}\cos^2\theta}}$$
        Finally, seen from the particle's frame the other particle is travelling in the
        opposite direction along the $y$ direction therefore
        \begin{align*}
            V &= \frac{u_y' + u_y'}{1 + u_y'^2/c^2}\\
              &=\frac{\frac{2v\sin\theta}{\sqrt{1 - \frac{v^2}{c^2}\cos^2\theta}}}
              {1 + \frac{v^2\sin^2\theta}{c^2 - v^2\cos^2\theta}}\\
              &= \frac{\frac{2v\sin\theta\sqrt{1 - \frac{v^2}{c^2}\cos^2\theta}}{1 - \frac{v^2}{c^2}\cos^2\theta}}
              {\frac{c^2 - v^2\cos^2\theta + v^2\sin^2\theta}{c^2(1 - \frac{v^2}{c^2}\cos^2\theta)}}\\
              &= \frac{2v\sin\theta\sqrt{1 - \frac{v^2}{c^2}\cos^2\theta}}
              {1 - \frac{v^2}{c^2}\cos 2\theta}
        \end{align*}
    \end{proof}
    \begin{proof}{\textbf{2.6}}
        \begin{itemize}
            \item [(a)] From $A$'s frame the ground is moving in westward as $B$ so by
            using the velocity addition formula we have that $A$ sees $B$ travelling
            with a velocity
            $$V = \frac{2v}{1 + v^2/c^2}$$
            Also from $A$'s point of view the train is length contracted by $2L/\gamma_V$.
            Where $\gamma_V$ is 
            \begin{align*}
                \gamma_V &= \frac{c}{\sqrt{c^2 - V^2}}
                    = \frac{c}{\sqrt{c^2 - (\frac{2v}{1 + v^2/c^2})^2}}\\
                    &= \frac{c}{\sqrt{c^2 - \frac{(2vc^2)^2}{(c^2 + v^2)^2}}}
                    = \frac{1}{\sqrt{1 - \frac{(2vc)^2}{(c^2 + v^2)^2}}}\\
                    &= \frac{1}{\sqrt{\frac{(c^2+v^2)^2 - (2vc)^2}{(c^2 + v^2)^2}}}
                    = \frac{c^2 + v^2}{\sqrt{((c^2)^2+2c^2v^2+(v^2)^2)^2 - (2vc)^2}}\\
                    &= \frac{c^2 + v^2}{c^2 -v^2}
                \end{align*}
            so the time it takes for the trains to pass is $(L + (2L/\gamma_V))/V$
            \begin{align*}
                \Delta t_A &= \frac{L + 2L(\frac{1 - v^2/c^2}{1 + v^2/c^2})}
                {\frac{2v}{1 + v^2/c^2}}\\
                \Delta t_A &= \frac{L((1 + v^2/c^2) + 2(1 - v^2/c^2))}{2v}\\
                \Delta t_A &= \frac{L(3 - v^2/c^2)}{2v}
            \end{align*}
            \item [(b)] From $B$'s frame the ground is moving in eastward as $A$ so by
            using the velocity addition formula we have that $B$ sees $A$ travelling
            with a velocity
            $$V = \frac{2v}{1 + v^2/c^2}$$
            Also from $B$'s point of view the train is length contracted by $L/\gamma_V$.
            Where $\gamma_V$ as before is 
            \begin{align*}
                \gamma_V &= \frac{c^2 + v^2}{c^2 -v^2}
                \end{align*}
            so the time it takes for the trains to pass is $(2L + (L/\gamma_V))/V$
            \begin{align*}
                \Delta t_B &= \frac{2L + L(\frac{1 - v^2/c^2}{1 + v^2/c^2})}
                {\frac{2v}{1 + v^2/c^2}}\\
                \Delta t_B &= \frac{L(2(1 + v^2/c^2) + (1 - v^2/c^2))}{2v}\\
                \Delta t_B &= \frac{L(3 + v^2/c^2)}{2v}
            \end{align*}
            \item [(c)] In the ground frame we see both trains approaching with velocity
            $2v$ but also we see them length contracted by $L/\gamma_v$ and $2L/\gamma_v$
            respectively then the time it takes for the trains to pass is given by
            \begin{align*}
                \Delta t_g &= \frac{L/\gamma_v + 2L/\gamma_v}{2v}\\
                \Delta t_g &= \frac{3L\sqrt{1 - v^2/c^2}}{2v}\\
            \end{align*}
            \item [(d)] Let us now check that all the invariant intervals are the same.
            In $A$'s frame we have that
            \begin{align*}
                c^2\Delta t_A^2 - \Delta x^2 &= \frac{c^2L^2(3 - v^2/c^2)^2}{(2v)^2} - L^2\\
                    &= L^2(\frac{9c^2 - 6v^2 + v^4/c^2}{(2v)^2} - 1)\\
                    &= \frac{L^2}{4v^2}(9c^2 - 10v^2 + v^4/c^2)
            \end{align*}
            In $B$'s frame we have that
            \begin{align*}
                c^2\Delta t_B^2 - \Delta x^2 &= \frac{c^2L^2(3 + v^2/c^2)^2}{(2v)^2} - 4L^2\\
                    &= L^2(\frac{9c^2 + 6v^2 + v^4/c^2}{(2v)^2} - 4)\\
                    &= \frac{L^2}{4v^2}(9c^2 - 10v^2 + v^4/c^2)
            \end{align*}
            And finally in the ground's frame we have to determine first $\Delta x$
            since both trains travel with the same velocity in opposite direction the
            back of each train should meet in the middle of the entire length measured
            from the ground i.e. $(3L/\gamma_v)/2$ but also we need to subtract the
            length of train $A$ since we measure starting from the moment both fronts
            meet. Therefore 
            \begin{align*}
                c^2\Delta t_g^2 - \Delta x^2 &= \frac{c^23^2L^2(1 - v^2/c^2)}{(2v)^2} -
                    \left(\frac{L}{2}\sqrt{1-\frac{v^2}{c^2}}\right)^2\\
                    &= L^2\left(\frac{9c^2-9v^2}{(2v)^2} - 
                    (\frac{1}{4}-\frac{1}{4}\frac{v^2}{c^2})\right)\\
                    &= \frac{L^2}{4v^2}(9c^2 -10v^2 + \frac{v^4}{c^2})
            \end{align*}
            
        \end{itemize}       
    \end{proof}
    \begin{proof}{\textbf{2.8}}
        We know that the first event happens at $(x,t) = (0,0)$ and the the next event
        happens at $(x,t) = (2,1)$. Let us have another frame $S'$ where the
        worldline $x'$ passes through the point $(x,t) = (2,1)$ which implies that the
        two events happen simultaneously in $S'$. The angle between $x$ and $x'$ is
        given by
        $$\tan\theta = 1/2 = \beta = v/c$$
        Therefore our new frame $S'$ must be travelling with a velocity $v = c/2$.
         
    \end{proof}
    \begin{proof}{\textbf{2.9}}
        All the points in the hyperbola must satisfy that $c^2t^2 - x^2= 1$ and because
        of the invariant interval the points should also satisfy that
        $c^2t'^2 - x'^2= 1$. This means that the point where the $ct'$ line encounters
        the hyperbola should also satisfy this constraint that we know is a point where
        $x' = 0$ so this is a point where $ct' = 1$ (a unit value).
        
        On the other hand, we have that $\tan \theta_1 = \beta = x/ct$ therefore
        $x= \beta \cdot ct$ and replacing this value in the hyperbola equation we get
        that
        \begin{align*}
            c^2t^2 - \beta^2c^2t^2 &= 1\\
            ct\sqrt{1 - \beta^2} &= 1
        \end{align*} 
        Also, we have that the point where the hyperbola joins the line $ct'$ (a unit
        value on $ct'$) is at a distance
        $\sqrt{c^2t^2 + x^2} = \sqrt{c^2t^2 + (\beta ct)^2}$ from the origin,
        then we have that 
        \begin{align*}
            \sqrt{c^2t^2 + (\beta ct)^2} &= ct\sqrt{1 + \beta^2} 
            = \sqrt{\frac{1 + \beta^2}{1 - \beta^2}}
        \end{align*}
        Therefore this is the length on the paper that one unit on the $ct'$ axis has
        and also since we are taking that one unit on the $ct$ has a length of 1 in the
        paper this is also the ratio between one unit in the $ct'$ axis and one
        unit on the $ct$ axis.
    \end{proof}
	\begin{proof}{\textbf{2.13}}
        An observer on the initial inertial frame will see $d$ as the string
        length which is the length contracted proper distance $d'$ seen from the
        spaceships i.e.
        $$d = d'/\gamma$$
        Then, this means that the proper length of the string we see from a spaceship is
        $d' = \gamma d$. Since $\gamma = 1/\sqrt{1 - v^2/c^2}$ when the velocity
        increase due to the acceleration, $\gamma$ increases, therefore, $d'$
        measured from a spaceship increases which means that the string ends up breaking.
    \end{proof}
	\begin{proof}{\textbf{2.14}}
        Let us have two Lorentz transformations where in the first one we transform $x$
        and $ct$ to $x'$ and $ct'$ assuming $v_1 = \tanh\phi_1$ and in the second one
        we transform $x'$ and $ct'$ to $x''$ and $ct''$ assuming $v_2 = \tanh\phi_2$ 
        then we have the following set of equations.
        \begin{align*}
            x &= x'\cosh\phi_1 + ct'\sinh\phi_1\\
            ct &= x'\sinh\phi_1 + ct'\cosh\phi_1
        \end{align*}
        And for the second transformation, we have that
        \begin{align*}
            x' &= x''\cosh\phi_2 + ct''\sinh\phi_2\\
            ct' &= x''\sinh\phi_2 + ct''\cosh\phi_2
        \end{align*}
        Then if we want to transform directly from $x$ to $x''$ we replace variables
        such that we have
        \begin{align*}
            x &= (x''\cosh\phi_2 + ct''\sinh\phi_2)\cosh\phi_1 +
            (x''\sinh\phi_2 + ct''\cosh\phi_2)\sinh\phi_1\\
            x &= x''\cosh\phi_2\cosh\phi_1 + ct''\sinh\phi_2\cosh\phi_1 +
            x''\sinh\phi_2\sinh\phi_1 + ct''\cosh\phi_2\sinh\phi_1\\
            x &= x''(\cosh\phi_2\cosh\phi_1 + \sinh\phi_2\sinh\phi_1) +
            ct''(\sinh\phi_2\cosh\phi_1 + \cosh\phi_2\sinh\phi_1)\\
            x &= x''\cosh(\phi_1 + \phi_2) + ct''\sinh(\phi_1 + \phi_2)
        \end{align*}
        Where we used that $\cosh(\phi_1 + \phi_2) = \cosh\phi_2\cosh\phi_1 + \sinh\phi_2\sinh\phi_1$
        and that $\sinh(\phi_1 + \phi_2) = \sinh\phi_2\cosh\phi_1 + \cosh\phi_2\sinh\phi_1$

        In the same way, we calculate the same to transform from $ct$ to $ct''$ as
        follows
        \begin{align*}
            ct &= (x''\cosh\phi_2 + ct''\sinh\phi_2)\sinh\phi_1 +
            (x''\sinh\phi_2 + ct''\cosh\phi_2)\cosh\phi_1\\
            ct &= x''\cosh\phi_2\sinh\phi_1 + ct''\sinh\phi_2\sinh\phi_1 +
            x''\sinh\phi_2\cosh\phi_1 + ct''\cosh\phi_2\cosh\phi_1\\
            ct &= x''(\cosh\phi_2\sinh\phi_1 + \sinh\phi_2\cosh\phi_1) +
            ct''(\sinh\phi_2\sinh\phi_1 + \cosh\phi_2\cosh\phi_1)\\
            ct &= x''\sinh(\phi_1 + \phi_2) + ct''\cosh(\phi_1 + \phi_2)
        \end{align*}
        Therefore we see that applying twice the Lorentz transformation gives us the same
        result as applying one Lorentz transformation with a velocity\\
        $v = \tanh(\phi_1 + \phi_2)$.
    \end{proof}
\cleardoublepage
	\begin{proof}{\textbf{2.15}}
        From the equation $2.53$ we know that the velocity in terms of the time $t'$ of
        the spaceship is given by
        \begin{align*}
            \beta(t') = \tanh(at'/c)
        \end{align*}
        And we know that in the lab frame we see a clock in the spaceship running slow
        so we have that
        \begin{align*}
            dt &= \gamma dt' = \frac{dt'}{\sqrt{1-\beta^2}}\\
            \int_0^t dt &= \int_0^{t'} \frac{dt'}{\sqrt{1-\beta^2}} \\
            t &= \int_0^{t'} \cosh(at'/c)~dt'\\
            t &= \frac{c}{a}\sinh\left(\frac{at'}{c}\right)\\
        \end{align*}
        Where we used that $1/\sqrt{1-\tanh(x)^2} = \cosh(x)$.
    \end{proof}



\cleardoublepage
	\begin{proof}{\textbf{2.16}}
        Let us have two events which are the photon passing through the origin of
        $S'$ and another one which is the photon passing through a point $A$ at a distance
        $\Delta x'$ from the origin. Between this two events $\Delta t'$ time passes 
        and we know the velocity of the photon must be $c$ in $S'$ so we have that
        $$\frac{\Delta x'}{\Delta t'} = c$$
        Now by the Lorentz Transformations we can compute $\Delta x$ and 
        $\Delta t$ for the $S$ frame but we are interested in the velocity of the
        photon in the $S$ frame so we actually want to calculate 
        $\frac{\Delta x}{\Delta t}$ then we have that
        \begin{align*}
            \frac{\Delta x}{\Delta t} &= \frac{\gamma(\Delta x' + v\Delta t')}{\gamma(\Delta t' + v\Delta x'/c^2)}\\
                &= \frac{\Delta x'(1 + v\frac{\Delta t'}{\Delta x'})}{\Delta t'(1 + \frac{v\Delta x'}{c^2\Delta t'})}\\
                &= \frac{\Delta x'(1 + \frac{v}{c})}{\Delta t'(1 + \frac{v}{c})}\\
                &= \frac{\Delta x'}{\Delta t'} = c
        \end{align*} 
    \end{proof}
	\begin{proof}{\textbf{2.17}}
        Let us suppose we have a set of transformations from $S'$ to $S$ as follows
        \begin{align*}
            \Delta x &= \gamma_{v_1} (\Delta x' + v_1\Delta t')\\
            \Delta t &= \gamma_{v_1} (\Delta t' + v_1\Delta x'/c^2)
        \end{align*}
        And another set from $S''$ to $S'$
        \begin{align*}
            \Delta x' &= \gamma_{v_2} (\Delta x'' + v_2\Delta t'')\\
            \Delta t' &= \gamma_{v_2} (\Delta t'' + v_2\Delta x''/c^2)
        \end{align*}
        We want to find a set of transformations from $S''$ to $S$ then by replacing
        values we have that
        \begin{align*}
            \Delta x &= \gamma_{v_1} (\gamma_{v_2} (\Delta x'' + v_2\Delta t'')
                + v_1(\gamma_{v_2} (\Delta t'' + v_2\Delta x''/c^2)))\\
                &= \gamma_{v_1} \gamma_{v_2}(\Delta x'' + v_2\Delta t''
                + v_1\Delta t'' + v_1v_2\Delta x''/c^2))\\
                &= \gamma_{v_1} \gamma_{v_2}(\Delta x''(1 + \frac{v_1v_2}{c^2}) + 
                \Delta t'' (v_1 + v_2))\\
                &= \gamma_{v_1} \gamma_{v_2}(1 + \frac{v_1v_2}{c^2})(\Delta x''
                + \frac{v_1 + v_2}{1 + v_1v_2/c^2}\Delta t'')
        \end{align*}
        Let us now verify that if $v_3 = \frac{v_1 + v_2}{1 + v_1v_2/c^2}$ then 
        $\gamma_{v_3} = \gamma_{v_1} \gamma_{v_2}(1 + \frac{v_1v_2}{c^2})$ as follows
        \begin{align*}
            \gamma_{v_3} &= \frac{1 + \frac{v_1v_2}{c^2}}{
                \sqrt{1-\frac{v_1^2}{c^2}}\sqrt{1-\frac{v_2^2}{c^2}}}\\
                &= \sqrt{\frac{(1 + \frac{v_1v_2}{c^2})^2}{
                (1-\frac{v_1^2}{c^2})(1-\frac{v_2^2}{c^2})}}\\
                &= \sqrt\frac{1 + \frac{2v_1v_2}{c^2} + \frac{v_1^2v_2^2}{c^4}}{
                1 - \frac{v_1^2}{c^2} - \frac{v_2^2}{c^2} + \frac{v_2^2v_1^2}{c^4}}\\
                &= \sqrt\frac{c^2 + 2v_1v_2 + \frac{v_1^2v_2^2}{c^2}}{
                c^2 - v_1^2 - v_2^2 + \frac{v_2^2v_1^2}{c^2}}\\
                &= \sqrt\frac{1}{\frac{c^2 - v_1^2 +2v_1v_2 -2v_1v_2- v_2^2 + \frac{v_2^2v_1^2}{c^2}}{
                c^2 + 2v_1v_2 + \frac{v_1^2v_2^2}{c^2}}}\\
                &= \sqrt\frac{1}{1 - \frac{v_1^2 +2v_1v_2+ v_2^2}{
                c^2 + 2v_1v_2 + \frac{v_1^2v_2^2}{c^2}}}\\
                &= \sqrt\frac{1}{1 - \frac{(v_1 + v_2)^2}{
                c^2(1 + 2v_1v_2/c^2 + v_1^2v_2^2/c^4)}}\\
                &= \sqrt\frac{1}{1 - \frac{(v_1 + v_2)^2}{
                c^2(1 + v_1v_2/c^2)^2}} = \frac{1}{\sqrt{1 - \frac{v_3^2}{c^2}}}\\
        \end{align*}
        What's left is finding the transformation for $\Delta t$ so by replacing we
        have that
        \begin{align*}
            \Delta t &= \gamma_{v_1} (\gamma_{v_2} (\Delta t'' + v_2\Delta x''/c^2) +
             v_1(\gamma_{v_2} (\Delta x'' + v_2\Delta t''))/c^2)\\
             &= \gamma_{v_1} \gamma_{v_2} (\Delta t'' + v_2\Delta x''/c^2 +
             v_1\Delta x''/c^2 + v_1v_2\Delta t''/c^2)\\
             &= \gamma_{v_1} \gamma_{v_2} (\Delta t''(1 + v_1v_2/c^2) +
             (v_2 + v_1)/c^2 \Delta x'')\\
             &= \gamma_{v_1} \gamma_{v_2} (1 + v_1v_2/c^2) (\Delta t''
             + \frac{v_1+v_2}{1 + v_1v_2/c^2}\frac{\Delta x''}{c^2})\\
             &= \gamma_{v_3}(\Delta t'' + \frac{v_3}{c^2}\Delta x'')
        \end{align*}
    \end{proof}
\cleardoublepage
	\begin{proof}{\textbf{2.18}}
        \begin{itemize}
            \item[(a)]
            We want to compute the distance between the two events from the ground
            using the Lorentz Transformations then if $L'$ is the distance seen from the
            gound frame we have that
            $$L' = \gamma(L + vt)$$
            and since we have that $t = 0$ then we have that $L'= \gamma L$

            Now in the ground frame we have by the Lorentz Transformations that 
            $$t' = \gamma (t + vL/c^2)$$
            and since the measurement in the train frame happen simultaneously we have
            that $t = 0$ and therefore $t' = \gamma vL/c^2$
            \item [(b)]
            About the time measured between the two events on the ground frame
            let us assume that both events have a clock attached to it and both 
            happen when $t=0$ in the train frame but
            since this two events are not simultaneous in the ground frame, because of
            the rear clock ahead effect we have that when the left event happen at
            $t' = 0$ in the ground frame the right clock is showing $-Lv/c^2$ time,
            so the events will happen with a time interval of $Lv/c^2$ and also we see
            this clock running slower by a factor $\gamma$ then the real time between
            events is $\gamma Lv/c^2$.

            About the length we see from the ground frame we have that the length of the
            train is length contracted and as we know the two events are not
            simultaneous on the ground frame then we need to add the distance that
            accounts for this, so we have that
            \begin{align*}
                L' &= L/\gamma + \gamma Lv^2/c^2\\
                    &= \gamma L (\frac{1}{\gamma^2} + \frac{v^2}{c^2})\\
                    &= \gamma L ((1 - \frac{v^2}{c^2}) + \frac{v^2}{c^2})\\
                    &= \gamma L
            \end{align*}
            
            For the train frame, let us suppose that there are two people on the ground
            and the train sees them at each event, simultaneous for the train then
            the distance the train sees between the two people is $L = L' / \gamma$
            because of the length contraction which means that $L' = \gamma L$

            Now about the time we have that in the train frame both events happen
            simultaneously at $t=0$ but if we see the watch of the person on the right
            (the ground is moving leftward from the train perspective)
            at the moment the two events happen in the train frame we see that the watch
            is showing $Lv/c^2$ because of the rear clock ahead effect and in addition
            to that we see the watch moving slower because of the time dilation effect
            therefore the time between the two events in the ground frame as seen
            from the train frame is $\gamma Lv/c^2$.
        \end{itemize}
    \end{proof}
	\begin{proof}{\textbf{2.19}}
        Let $S'$ be the frame in which the clock moves vertically, so seen from our
        frame, $S$, the frame $S'$ is moving horizontally and therefore from the velocity
        addition formula in the y direction we have that
        $$u_y = \frac{u}{\gamma_{v}(1 + 0)} = \frac{u}{\gamma_v}$$
        Where we used that $u_y' = u$ and that $u_x' = 0$. Then we have that
        \begin{align*}
            u_y = \frac{\Delta y}{\Delta t} &= \frac{u}{\gamma_v}\\
            \Delta t &= \gamma_v \frac{\Delta y}{u}\\
            \Delta t &= \gamma_v \gamma_u \frac{\Delta y'}{u}\\
            \Delta t &= \gamma_v \gamma_u \Delta t'\\
        \end{align*}
        Here we used that $\Delta y = \Delta y' \gamma_u$ because of the length
        contraction and that $\Delta t' = \Delta y' / u$. Therefore the time we see
        is slowed by a factor $\gamma_v \gamma_u$.
    \end{proof}
	\begin{proof}{\textbf{2.20}}
        Let $S'$ be the train frame, and let $S$ be the ground frame. Then we can
        calculate the x and y photon's velocities seen from $S$ using the velocity
        addition formulas as follows
        \begin{align*}
            u_x = \frac{c/\sqrt{2} - c/\sqrt{2}}{1 - c^2/2c^2} = 0
        \end{align*}
        Here we used that $u_x' = -c/\sqrt{2}$ because the photon is moving leftward and
        $v = c/\sqrt{2}$ because $S'$ is moving rightward. And for the y component we
        have that
        \begin{align*}
            u_y &= \frac{c/\sqrt{2}}{\gamma_{c/\sqrt{2}}(1 - c^2/2c^2)}\\
                &= \frac{2c}{\sqrt{2} \gamma_{c/\sqrt{2}}}\\
                &= \frac{2c\sqrt{1-c^2/2c^2}}{\sqrt{2}}\\
                &= \frac{2c}{\sqrt{2}\sqrt{2}} = c
        \end{align*}
        Finally, these results make sense because given that the train is travelling the
        opposite way we see that the photon is travelling only vertically i.e. $u_x = 0$.
        Also, the velocity we see in the vertical direction is $c$ which is coherent
        with the fact that the velocity of the light should be the same in any reference
        frame.    
    \end{proof}
	\begin{proof}{\textbf{2.22}}
        \begin{itemize}
            \item [(i)] For the person $A$ standing on the ground the velocity between
            the train and the person is $c/5$ since we already have both velocities
            measured in the ground frame. Since we see the length of the train length
            contracted then the distance between the two events is $L/\gamma$. Then
            the time between the two events is
            $$\Delta t_A = \frac{5L}{c\gamma_{3c/5}} = \frac{4L}{c}$$
            Now the train $B$ will travel for a time $4L/c$ with a velocity of
            $3c/5$ so it will cover a distance of $12L/5$, but this measurement
            correspond to the distance travelled by the front of the train so we need to
            add the length of the train (taking into account the length contraction
            effect) i.e. $12L/5 + 4L/5 = 16L/5$. Therefore the invariant interval has a
            value of
            \begin{align*}
                c^2\Delta t_A^2 - \Delta x_A^2 = (4L)^2 - \frac{(16L)^2}{5^2}
                =  \frac{144}{25}L^2
            \end{align*}
            \item [(ii)] For the train $B$ the ground is travelling to the left (opposite
            to $C$) so by using the velocity addition formula we can calculate the
            velocity at which $C$ is moving seen from $B$ i.e.
            \begin{align*}
                V = \frac{4c/5 - 3c/5}{1 - (3\cdot 4)/5^2} = \frac{5}{13}c 
            \end{align*}
            From here we can calculate the time between the two events as follows
            \begin{align*}
                \Delta t_B = \frac{L}{\frac{5}{13}c} = \frac{13L}{5c}
            \end{align*}
            Therefore the invariant interval is
            \begin{align*}
                c^2\Delta t_B^2 - \Delta x_B^2 &= (\frac{13L}{5})^2 - L^2\\
                    &= L^2(\frac{169}{25} -1)\\
                    &= \frac{144 L^2}{25}
            \end{align*}
            \item [(iii)] For the person $C$ the ground is travelling to the left
            (opposite to $B$) so by using the velocity addition formula we can calculate
            the velocity at which $B$ is moving seen from $C$ i.e.
            \begin{align*}
                V = \frac{3c/5 - 4c/5}{1 - (3\cdot 4)/5^2} = -\frac{5}{13}c 
            \end{align*}
            From here we can calculate the time between the two events as follows
            \begin{align*}
                \Delta t_C = \frac{L/\gamma_V}{\frac{5}{13}c} = \frac{12L}{5c}
            \end{align*}
            Where we used that $\gamma_V = \gamma_{5c/13} = 13/12$. Therefore the
            invariant interval is
            \begin{align*}
                c^2\Delta t_C^2 - \Delta x_C^2 &= (\frac{12L}{5})^2 - 0\\
                    &= L^2(\frac{144}{25})
            \end{align*}
            In this invariant interval the distance between the two events is 0 since
            both happen at $C$ from $C$'s frame.
        \end{itemize}
    \end{proof}
	\begin{proof}{\textbf{2.25}}
        \begin{itemize}
        \item [(a)] In the barn frame the events happen at a distance $\Delta x = L$.
        To compute the time between events we must take into account that the first
        event (seen from the barn frame) happens at the back of the pole and the other
        event happens at the front of the pole, therefore we subract the distance of the
        barn $L$ from the distance of the pole which is length contracted to $L/\gamma$
        then we calculate the following
        \begin{align*}
            \Delta t &= \frac{L - L/\gamma_{3/5}}{3c/5}\\
                &= \frac{L/5}{3c/5}\\
                &= \frac{L}{3c}
        \end{align*}
        \item [(b)] In the Minkowsky diagram we see that $E_2$ being "the front of the
        pole passing the right end of the barn" is happening before $E_1$ being
        "the back of the pole passing the left end of the barn" seen from the pole
        frame, then we can find a frame where the
        $\Delta x'$ axis passes through $E_2$ with coordinates $(L, L/3)$, meaning
        that $E_1$ and $E_2$ would happen simultaneously from this frame, then our
        new $\Delta x'$ axis must have an inclination of 
        \begin{align*}
            \tan \theta = \frac{c\Delta t}{\Delta x} = \frac{L/3}{L} = \frac{1}{3}
        \end{align*}
        The axis in the pole frame has an inclination of
        $\theta = \arctan(3/5) = 30.96^{\circ}$
        which is bigger than the inclination we get for the axis in this new frame
        $\theta = \arctan(1/3) = 18.43^{\circ}$ as we wanted.
        \end{itemize}
    \end{proof}
\cleardoublepage
	\begin{proof}{\textbf{2.30}}
        \begin{itemize}
        \item [(a)] If we imagine a stationary observer C between A and B, C will see
        the flashes from A with a frequency given by the equation we have
        \begin{align*}
            f_C = \sqrt{\frac{1+\beta}{1-\beta}}~f_A
        \end{align*} 
        where in this case $f_A$ is the frequency A says he is sending the flashes from
        its frame. But since C is re-sending the flashes instantaneously then B will
        see the flashes with a frequency of
        $$f_B = \sqrt{\frac{1+\beta}{1-\beta}}~f_C$$ 
        Therefore in terms of $f_A$ we have that the frequency that B sees is given by
        $$f_B = \frac{1+\beta}{1-\beta}~f_A$$

        \item [(b)] In this case B sees A travelling at a velocity $V$ given by the 
        relativistic sum of velocities i.e.
        \begin{align*}
            V = \frac{2v}{1+v^2/c^2}
        \end{align*}
        Then from the frequency equation which applies in this case too, we have that
        \begin{align*}
            f_B &= \sqrt{\frac{1 + \frac{V}{c}}{1 - \frac{V}{c}}}~f_A\\
            f_B &= \sqrt{\frac{1 + \frac{2v}{c(1+v^2/c^2)}}{1 - \frac{2v}{c(1+v^2/c^2)}}}~f_A\\
            f_B &= \sqrt{\frac{\frac{c(1+v^2/c^2 + 2v/c)}{c(1+v^2/c^2)}}
            {\frac{c(1+v^2/c^2 - 2v/c)}{c(1+v^2/c^2)}}}~f_A\\
            f_B &= \sqrt{\frac{(1+v/c)^2}{(1-v/c)^2}}~f_A\\
            f_B &= \frac{1+\beta}{1 - \beta}~f_A
        \end{align*}
        Which is the same result we obtained the other way.
        \end{itemize}
    \end{proof}
\cleardoublepage
	\begin{proof}{\textbf{2.34}}
        \begin{itemize}
            \item [(a)] From the time equation we got from (2.15) and the velocity
            equation (2.53) we have that
            \begin{align*}
                t' &= \frac{c}{a}\sinh^{-1}\left(\frac{at}{c}\right)\\
                v(t') &= c\tanh(\frac{at'}{c})
            \end{align*}
            By replacing variables we have that
            \begin{align*}
                v(t) &= c\tanh(\sinh^{-1}\left(\frac{at}{c}\right))\\
                v(t) &= \frac{at}{\sqrt{(at/c)^2 + 1}}
            \end{align*}
            Since we know $v(t) = dL/dt$ we can integrate to obtain $L$ as a function
            of $t$ as follows
            \begin{align*}
                L &= c\int_0^t \frac{at/c}{\sqrt{(at/c)^2 + 1}} dt\\
                L &= c\left[\frac{c}{a}\sqrt{\left(\frac{at}{c}\right)^2 + 1} - \frac{c}{a}\right]\\
                \left(\frac{aL}{c^2} + 1\right)^2 - 1 &= \left(\frac{at}{c}\right)^2 \\
                \left(\frac{at}{c}\right)^2 &= \frac{a^2L^2}{c^4} + \frac{2aL}{c^2}\\
                t^2 &= \frac{c^2}{a^2}\left(\frac{a^2L^2}{c^4} + \frac{2aL}{c^2}\right) \\
                t &= \sqrt{\frac{L^2}{c^2} + \frac{2L}{a}}\\
                t &= \frac{L}{c}\sqrt{1 + \frac{2c^2/a}{L}}
            \end{align*}
            If $L \to 0$ then $L^2/c^2$ is negligible so we have that 
            \begin{align*}
                t &= \sqrt{\frac{2L}{a}}\\
                L &=\frac{1}{2}at^2
            \end{align*}
            Where we have here the known formula for $L$ used in the non-relativistic
            theory.\\
            If $L \to \infty$ we have that $\frac{2c^2/a}{L}$ is negligible then we get
            $$t = \frac{L}{c}$$
            \item [(b)] From part (a) we have an expression of $t'$ in terms of $t$ so
            we can replace this value with the expression we found in terms of $L$
            as follows
            \begin{align*}
                t' &= \frac{c}{a}\sinh^{-1}\left(\frac{at}{c}\right)\\
                t' &= \frac{c}{a}\sinh^{-1}\left(\frac{a}{c}\sqrt{\frac{L^2}{c^2} + \frac{2L}{a}}\right)
                % L' &= \int_0^{t'} v(t')~dt'\\
                % L' &= \int_0^{t'} c\tanh(\frac{at'}{c})~dt'\\
                % L' &= \frac{c^2}{a}\log(\cosh(\frac{at'}{c}))\\
                % \frac{at'}{c} &= \cosh^{-1}(e^{\frac{aL'}{c^2}})\\
                % t' &= \frac{c}{a}\cosh^{-1}(e^{\frac{aL'}{c^2}})
            \end{align*}
            If $L \to 0$ then $L^2/c^2$ is negligible so we have that
            \begin{align*}
                t' &= \frac{c}{a}\sinh^{-1}\left(\sqrt{\frac{2La}{c^2}}\right)\\
                t' &= \frac{c}{a}\sqrt{\frac{2La}{c^2}}\\
                t' &= \sqrt{\frac{2L}{a}}\\
                L &= \frac{1}{2}a{t'}^{2}
            \end{align*}
            Here we used the fact that $\sinh^{-1}(x) \approx x$ when $x$ is small, and
            we obtained the formula for $L$ used in non-relativistic theory again.\\
            If $L \to \infty$ we have that $\frac{2c^2/a}{L}$ is negligible then we get
            that
            $$t' = \frac{c}{a}\sinh^{-1}\left(\frac{aL}{c^2}\right)$$
        \end{itemize}
    \end{proof}


\end{document}























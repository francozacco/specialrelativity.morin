\documentclass[11pt]{article}
\usepackage{amssymb}
\usepackage{amsthm}
\usepackage{enumitem}
\usepackage{amsmath}
\usepackage{bm}
\usepackage{adjustbox}
\usepackage{mathrsfs}
\usepackage{graphicx}
\usepackage{siunitx}
\usepackage[mathscr]{euscript}

\title{\textbf{Solved selected problems of Special Relativity - Morin}}
\author{Franco Zacco}
\date{}

\addtolength{\topmargin}{-3cm}
\addtolength{\textheight}{3cm}

\newcommand{\hatr}{\bm{\hat{r}}}
\newcommand{\hatx}{\bm{\hat{x}}}
\newcommand{\haty}{\bm{\hat{y}}}
\newcommand{\hatz}{\bm{\hat{z}}}
\newcommand{\hatth}{\bm{\hat{\theta}}}
\newcommand{\hatphi}{\bm{\hat{\phi}}}
\newcommand{\hatrho}{\bm{\hat{\rho}}}
\theoremstyle{definition}
\newtheorem*{solution*}{Solution}

\begin{document}
\maketitle
\thispagestyle{empty}

\section*{Chapter 2 - Kinematics, Part 2}

	\begin{proof}{\textbf{2.1}}
        By checking the Excercise 1 of Section 1.5 we have that the velocity $v$ and
        the factor $\gamma$ for the different frame combinations are
        \begin{table}[h]
            \centering
            \begin{tabular}{lllllll}
            \hline
                     & AB      & AC     & AD            & BC     & BD            & CD            \\ \hline
            $v$      & $5c/13$ & $4c/5$ & $c/5$         & $3c/5$ & $c/5$         & $5c/7$        \\
            $\gamma$ & $13/12$ & $5/3$  & $5/2\sqrt{6}$ & $5/4$  & $5/2\sqrt{6}$ & $7/2\sqrt{6}$ \\
            \end{tabular}
        \end{table}

        Now let us verify the values of the table we have using the Lorentz
        Transformations
        \begin{align*}
            \Delta x &= \gamma (\Delta x' + v\Delta t')\\
            \Delta t &= \gamma (\Delta t' + v\Delta x'/c^2)
        \end{align*}
        From frame $A$ with $\Delta t_A$ and $\Delta x_A$ we calculate $\Delta t_B$ and
        $\Delta x_B$ for frame $B$ as follows
        \begin{align*}
            \Delta x_B &= \gamma_{AB}(\Delta x_A + v_{AB}\Delta t_A)\\
                &= \frac{13}{12}(-L + \frac{5c}{13}\frac{5L}{c}) \\
                &= \frac{13}{12}(\frac{12}{13}L) = L \\
            \Delta t_B &= \gamma_{AB} (\Delta t_A + v_{AB}\Delta x_A/c^2)\\
                &= \frac{13}{12}(\frac{5L}{c} - \frac{5c}{13}\frac{L}{c^2}) \\
                &= \frac{13}{12}(\frac{60}{13}\frac{L}{c}) = \frac{5L}{c}
        \end{align*}
\cleardoublepage
        From frame $A$ with $\Delta t_A$ and $\Delta x_A$ we calculate $\Delta t_C$ and
        $\Delta x_C$ for frame $C$ as follows
        \begin{align*}
            \Delta x_C &= \gamma_{AC}(\Delta x_A + v_{AC}\Delta t_A)\\
                &= \frac{5}{3}(-L + \frac{4c}{5}\frac{5L}{c}) \\
                &= \frac{5}{3}(3L) = 5L \\
            \Delta t_C &= \gamma_{AC} (\Delta t_A + v_{AC}\Delta x_A/c^2)\\
                &= \frac{5}{3}(\frac{5L}{c} - \frac{4c}{5}\frac{L}{c^2}) \\
                &= \frac{5}{3}(\frac{21}{3}\frac{L}{c}) = \frac{7L}{c}
        \end{align*}
        From frame $A$ with $\Delta t_A$ and $\Delta x_A$ we calculate $\Delta t_D$ and
        $\Delta x_D$ for frame $D$ as follows
        \begin{align*}
            \Delta x_D &= \gamma_{AD}(\Delta x_A + v_{AD}\Delta t_A)\\
                &= \frac{5}{2\sqrt{6}}(-L + \frac{c}{5}\frac{5L}{c}) \\
                &= \frac{5}{2\sqrt{6}} \cdot 0 = 0 \\
            \Delta t_D &= \gamma_{AD} (\Delta t_A + v_{AD}\Delta x_A/c^2)\\
                &= \frac{5}{2\sqrt{6}}(\frac{5L}{c} - \frac{c}{5}\frac{L}{c^2}) \\
                &= \frac{5}{2\sqrt{6}}(\frac{24}{5}\frac{L}{c}) = \frac{2\sqrt{6}L}{c}
        \end{align*}
        From frame $B$ with $\Delta t_B$ and $\Delta x_B$ we calculate $\Delta t_C$ and
        $\Delta x_C$ for frame $C$ as follows
        \begin{align*}
            \Delta x_C &= \gamma_{BC}(\Delta x_B + v_{BC}\Delta t_B)\\
                &= \frac{5}{4}(L + \frac{3c}{5}\frac{5L}{c}) \\
                &= \frac{5}{4}(4L) = 5L \\
            \Delta t_C &= \gamma_{BC} (\Delta t_B + v_{BC}\Delta x_B/c^2)\\
                &= \frac{5}{4}(\frac{5L}{c} + \frac{3c}{5}\frac{L}{c^2}) \\
                &= \frac{5}{4}(\frac{28}{5}\frac{L}{c}) = \frac{7L}{c}
        \end{align*}
\cleardoublepage
        From frame $B$ with $\Delta t_B$ and $\Delta x_B$ we calculate $\Delta t_D$ and
        $\Delta x_D$ for frame $D$ taking into account that in this case $B$ moves
        to the left so the sign of the second term is negative then 
        \begin{align*}
            \Delta x_D &= \gamma_{BD}(\Delta x_B - v_{BD}\Delta t_B)\\
                &= \frac{5}{2\sqrt{6}}(L - \frac{c}{5}\frac{5L}{c}) \\
                &= \frac{5}{2\sqrt{6}} \cdot 0 = 0 \\
            \Delta t_D &= \gamma_{BD} (\Delta t_B - v_{BD}\Delta x_B/c^2)\\
                &= \frac{5}{2\sqrt{6}}(\frac{5L}{c} - \frac{c}{5}\frac{L}{c^2}) \\
                &= \frac{5}{2\sqrt{6}}(\frac{24}{5}\frac{L}{c}) = \frac{2\sqrt{6}L}{c}
        \end{align*}
        From frame $C$ with $\Delta t_C$ and $\Delta x_C$ we calculate $\Delta t_D$ and
        $\Delta x_D$ for frame $D$ taking into account that in this case $C$ moves
        to the left so the sign of the second term is negative then 
        \begin{align*}
            \Delta x_D &= \gamma_{CD}(\Delta x_C - v_{CD}\Delta t_C)\\
                &= \frac{7}{2\sqrt{6}}(5L - \frac{5c}{7}\frac{7L}{c}) \\
                &= \frac{7}{2\sqrt{6}} \cdot 0 = 0 \\
            \Delta t_D &= \gamma_{CD} (\Delta t_C - v_{CD}\Delta x_C/c^2)\\
                &= \frac{7}{2\sqrt{6}}(\frac{7L}{c} - \frac{5c}{7}\frac{5L}{c^2}) \\
                &= \frac{7}{2\sqrt{6}}(\frac{24}{7}\frac{L}{c}) = \frac{2\sqrt{6}L}{c}
        \end{align*}
    \end{proof}
\cleardoublepage
	\begin{proof}{\textbf{2.16}}
        Let us have two events which are the photon passing through the origin of
        $S'$ and another one which is the photon passing through a point $A$ at a distance
        $\Delta x'$ from the origin. Between this two events $\Delta t'$ time passes 
        and we know the velocity of the photon must be $c$ in $S'$ so we have that
        $$\frac{\Delta x'}{\Delta t'} = c$$
        Now by the Lorentz Transformations we can compute $\Delta x$ and 
        $\Delta t$ for the $S$ frame but we are interested in the velocity of the
        photon in the $S$ frame so we actually want to calculate 
        $\frac{\Delta x}{\Delta t}$ then we have that
        \begin{align*}
            \frac{\Delta x}{\Delta t} &= \frac{\gamma(\Delta x' + v\Delta t')}{\gamma(\Delta t' + v\Delta x'/c^2)}\\
                &= \frac{\Delta x'(1 + v\frac{\Delta t'}{\Delta x'})}{\Delta t'(1 + \frac{v\Delta x'}{c^2\Delta t'})}\\
                &= \frac{\Delta x'(1 + \frac{v}{c})}{\Delta t'(1 + \frac{v}{c})}\\
                &= \frac{\Delta x'}{\Delta t'} = c
        \end{align*} 
    \end{proof}
	\begin{proof}{\textbf{2.17}}
        Let us suppose we have a set of transformations from $S'$ to $S$ as follows
        \begin{align*}
            \Delta x &= \gamma_{v_1} (\Delta x' + v_1\Delta t')\\
            \Delta t &= \gamma_{v_1} (\Delta t' + v_1\Delta x'/c^2)
        \end{align*}
        And another set from $S''$ to $S'$
        \begin{align*}
            \Delta x' &= \gamma_{v_2} (\Delta x'' + v_2\Delta t'')\\
            \Delta t' &= \gamma_{v_2} (\Delta t'' + v_2\Delta x''/c^2)
        \end{align*}
        We want to find a set of transformations from $S''$ to $S$ then by replacing
        values we have that
        \begin{align*}
            \Delta x &= \gamma_{v_1} (\gamma_{v_2} (\Delta x'' + v_2\Delta t'')
                + v_1(\gamma_{v_2} (\Delta t'' + v_2\Delta x''/c^2)))\\
                &= \gamma_{v_1} \gamma_{v_2}(\Delta x'' + v_2\Delta t''
                + v_1\Delta t'' + v_1v_2\Delta x''/c^2))\\
                &= \gamma_{v_1} \gamma_{v_2}(\Delta x''(1 + \frac{v_1v_2}{c^2}) + 
                \Delta t'' (v_1 + v_2))\\
                &= \gamma_{v_1} \gamma_{v_2}(1 + \frac{v_1v_2}{c^2})(\Delta x''
                + \frac{v_1 + v_2}{1 + v_1v_2/c^2}\Delta t'')
        \end{align*}
        Let us now verify that if $v_3 = \frac{v_1 + v_2}{1 + v_1v_2/c^2}$ then 
        $\gamma_{v_3} = \gamma_{v_1} \gamma_{v_2}(1 + \frac{v_1v_2}{c^2})$ as follows
        \begin{align*}
            \gamma_{v_3} &= \frac{1 + \frac{v_1v_2}{c^2}}{
                \sqrt{1-\frac{v_1^2}{c^2}}\sqrt{1-\frac{v_2^2}{c^2}}}\\
                &= \sqrt{\frac{(1 + \frac{v_1v_2}{c^2})^2}{
                (1-\frac{v_1^2}{c^2})(1-\frac{v_2^2}{c^2})}}\\
                &= \sqrt\frac{1 + \frac{2v_1v_2}{c^2} + \frac{v_1^2v_2^2}{c^4}}{
                1 - \frac{v_1^2}{c^2} - \frac{v_2^2}{c^2} + \frac{v_2^2v_1^2}{c^4}}\\
                &= \sqrt\frac{c^2 + 2v_1v_2 + \frac{v_1^2v_2^2}{c^2}}{
                c^2 - v_1^2 - v_2^2 + \frac{v_2^2v_1^2}{c^2}}\\
                &= \sqrt\frac{1}{\frac{c^2 - v_1^2 +2v_1v_2 -2v_1v_2- v_2^2 + \frac{v_2^2v_1^2}{c^2}}{
                c^2 + 2v_1v_2 + \frac{v_1^2v_2^2}{c^2}}}\\
                &= \sqrt\frac{1}{1 - \frac{v_1^2 +2v_1v_2+ v_2^2}{
                c^2 + 2v_1v_2 + \frac{v_1^2v_2^2}{c^2}}}\\
                &= \sqrt\frac{1}{1 - \frac{(v_1 + v_2)^2}{
                c^2(1 + 2v_1v_2/c^2 + v_1^2v_2^2/c^4)}}\\
                &= \sqrt\frac{1}{1 - \frac{(v_1 + v_2)^2}{
                c^2(1 + v_1v_2/c^2)^2}} = \frac{1}{\sqrt{1 - \frac{v_3^2}{c^2}}}\\
        \end{align*}
        What's left is finding the transformation for $\Delta t$ so by replacing we
        have that
        \begin{align*}
            \Delta t &= \gamma_{v_1} (\gamma_{v_2} (\Delta t'' + v_2\Delta x''/c^2) +
             v_1(\gamma_{v_2} (\Delta x'' + v_2\Delta t''))/c^2)\\
             &= \gamma_{v_1} \gamma_{v_2} (\Delta t'' + v_2\Delta x''/c^2 +
             v_1\Delta x''/c^2 + v_1v_2\Delta t''/c^2)\\
             &= \gamma_{v_1} \gamma_{v_2} (\Delta t''(1 + v_1v_2/c^2) +
             (v_2 + v_1)/c^2 \Delta x'')\\
             &= \gamma_{v_1} \gamma_{v_2} (1 + v_1v_2/c^2) (\Delta t''
             + \frac{v_1+v_2}{1 + v_1v_2/c^2}\frac{\Delta x''}{c^2})\\
             &= \gamma_{v_3}(\Delta t'' + \frac{v_3}{c^2}\Delta x'')
        \end{align*}
    \end{proof}
\cleardoublepage
	\begin{proof}{\textbf{2.18}}
        \begin{itemize}
            \item[(a)]
            We want to compute the distance between the two events from the ground
            using the Lorentz Transformations then if $L'$ is the distance seen from the
            gound frame we have that
            $$L' = \gamma(L + vt)$$
            and since we have that $t = 0$ then we have that $L'= \gamma L$

            Now in the ground frame we have by the Lorentz Transformations that 
            $$t' = \gamma (t + vL/c^2)$$
            and since the measurement in the train frame happen simultaneously we have
            that $t = 0$ and therefore $t' = \gamma vL/c^2$
            \item [(b)]
            About the time measured between the two events on the ground frame
            let us assume that both events have a clock attached to it and both 
            happen when $t=0$ in the train frame but
            since this two events are not simultaneous in the ground frame, because of
            the rear clock ahead effect we have that when the left event happen at
            $t' = 0$ in the ground frame the right clock is showing $-Lv/c^2$ time,
            so the events will happen with a time interval of $Lv/c^2$ and also we see
            this clock running slower by a factor $\gamma$ then the real time between
            events is $\gamma Lv/c^2$.

            About the length we see from the ground frame we have that the length of the
            train is length contracted and as we know the two events are not
            simultaneous on the ground frame then we need to add the distance that
            accounts for this, so we have that
            \begin{align*}
                L' &= L/\gamma + \gamma Lv^2/c^2\\
                    &= \gamma L (\frac{1}{\gamma^2} + \frac{v^2}{c^2})\\
                    &= \gamma L ((1 - \frac{v^2}{c^2}) + \frac{v^2}{c^2})\\
                    &= \gamma L
            \end{align*}
            
            For the train frame, let us suppose that there are two people on the ground
            and the train sees them at each event, simultaneous for the train then
            the distance the train sees between the two people is $L = L' / \gamma$
            because of the length contraction which means that $L' = \gamma L$

            Now about the time we have that in the train frame both events happen
            simultaneously at $t=0$ but if we see the watch of the person on the right
            (the ground is moving leftward from the train perspective)
            at the moment the two events happen in the train frame we see that the watch
            is showing $Lv/c^2$ because of the rear clock ahead effect and in addition
            to that we see the watch moving slower because of the time dilation effect
            therefore the time between the two events in the ground frame as seen
            from the train frame is $\gamma Lv/c^2$.
        \end{itemize}
    \end{proof}

\end{document}























\documentclass[11pt]{article}
\usepackage{amssymb}
\usepackage{amsthm}
\usepackage{enumitem}
\usepackage{amsmath}
\usepackage{bm}
\usepackage{adjustbox}
\usepackage{mathrsfs}
\usepackage{graphicx}
\usepackage{siunitx}
\usepackage[mathscr]{euscript}

\title{\textbf{Solved selected problems of Special Relativity - Morin}}
\author{Franco Zacco}
\date{}

\addtolength{\topmargin}{-3cm}
\addtolength{\textheight}{3cm}

\newcommand{\hatr}{\bm{\hat{r}}}
\newcommand{\hatx}{\bm{\hat{x}}}
\newcommand{\haty}{\bm{\hat{y}}}
\newcommand{\hatz}{\bm{\hat{z}}}
\newcommand{\hatth}{\bm{\hat{\theta}}}
\newcommand{\hatphi}{\bm{\hat{\phi}}}
\newcommand{\hatrho}{\bm{\hat{\rho}}}
\theoremstyle{definition}
\newtheorem*{solution*}{Solution}
\renewcommand*{\proofname}{Solution}

\begin{document}
\maketitle
\thispagestyle{empty}

\section*{Chapter 1 - Kinematics, Part 1}

	\begin{proof}{\textbf{1.1}}
        The difference $t_R - t_L$ is given by
        $$t_R - t_L = \frac{l'}{c-v} - \frac{l'}{c+v}$$
        where $l'$ is half of the train length in the ground frame, but because
        of the length contraction we know that
        $$l' = \frac{L_{obs}}{2} = \frac{L_{prop}}{2\gamma}$$
        knowing that $\gamma = c/\sqrt{c^2 - v^2} = 1/\sqrt{1 - v^2/c^2}$ and
        calling $L = L_{prop}$ then
        \begin{align*}
            t_R - t_L &= \frac{L}{2\gamma(c-v)} - \frac{L}{2\gamma(c+v)}\\
                      &= \frac{L}{2\gamma}\left(\frac{1}{c-v}- \frac{1}{c+v}\right)\\
                      &= \frac{L}{2\gamma}\left(\frac{2v}{c^2-v^2}\right)\\
                      &= L\left(\frac{v\sqrt{c^2-v^2}}{c(c^2-v^2)}\right)\\
                      &= L\left(\frac{v}{c\sqrt{c^2-v^2}}\right)\\
                      &= \frac{Lv}{c^2\gamma}
        \end{align*}
        The right side of the train should be hit by the light with a time
        difference of $Lv/c^2$ from when the left clock was hitted (seen from
        an outside observer), but because it is a moving train the right side
        of the train is going to be hitted $Lv/c^2\gamma$ time after the left
        side because of the time dilation effect, which is what we got.   
    \end{proof}
\cleardoublepage
	\begin{proof}{\textbf{1.2}}
        For the person to be able to get to the back of the train before the
        clock on the back of the train hits $Lv/c^2$, the person should go at
        a velocity at least of 
        $$\frac{L}{Lv/c^2} = c^2/v$$
        which is over the speed of light and therefore it's impossible for an
        observer on the ground to see both the person at the front and at 
        back of the train. 
    \end{proof}
	\begin{proof}{\textbf{1.3}}
        In the $A$ and $B$ reference frame the time it takes $C$ to reach $B$
        is $L/v$ but $C$ runs slow by a factor $\gamma$ so only $L/\gamma v$
        elapses on $C$ during the journey. Therefore when $C$ reaches $B$ the
        reading on $C$ is $L/\gamma v$ and the reading on $B$ is $L/v$. If
        $v \ll c$ then we can use that $\sqrt{1 - \epsilon} \approx 1 - \epsilon/2$
        so
        $$\frac{L}{\gamma v} \approx \frac{L}{v}(1 - \frac{v^2}{2c^2}) =
        \frac{L}{v} - \frac{Lv}{2c^2}$$
        The difference between the readings of $C$ and $B$ is therefore
        $\frac{Lv}{2c^2}$ so the readings between $C$ and $B$ can be made
        arbitrarily close with a sufficiently small $v$.  
    \end{proof}
	\begin{proof}{\textbf{1.6}}
        From an observer at the barn frame the length of the pole is going to
        be contracted by a factor of $\gamma$, meaning that the observer will
        see the pole with a length $L/\gamma$ which is less than $L$, the
        proper length of the barn, so the pole for this observer is completely
        inside the barn.

        Let's see now what happens if the observer is now on the pole's frame,
        in the pole's frame the length is contracted so on the pole's frame the
        distance of the barn is $L/\gamma$ too, but this observer is going to see
        the opening and closing of the barn's doors not happening simultaneously.
        
        If the pole is travellling from left to right then from the pole's
        frame the barn is moving from right to left, so when the left door
        closes the right door is going to be still closed for a time $Lv/c^2$ also
        since the observer in the pole's frame sees this events time dilated,
        from the pole's frame the right door opens at time $\gamma Lv/c^2$.
        This means that the barn is going to travel an additional distance of
        $\gamma Lv^2/c^2$, then
        $$\frac{L}{\gamma} + \gamma L\frac{v^2}{c^2} = 
        \gamma L(\frac{1}{\gamma^2} + \frac{v^2}{c^2}) =
        \gamma L\left((1 - \frac{v^2}{c^2}) + \frac{v^2}{c^2}\right) = \gamma L$$
        Therefore, for an observer in the pole's frame the pole is going to be
        completely inside the barn too.
    \end{proof}
\cleardoublepage
	\begin{proof}{\textbf{1.7}}
        From the train frame the length of the tunnel is $L/\gamma$ because of
        the length contraction then the front of the train is going to hit
        first the front of the tunnel, provoking the train to explode.
        
        From the tunnel frame the length of the train is $L/\gamma$ because of
        the length contraction then the back of the train is going to hit first
        the back of the tunnel so the deactivation signal is sent to the bomb
        on the front but the signal travels with speed $c$ so it takes to reach
        the front of the train at the best a time $L/c$, so what we want is that
        at least
        \begin{align*}
            \frac{L}{c} &= \frac{L}{v} - \frac{L}{\gamma v}\\
            \frac{v}{c} &= 1 - \frac{1}{\gamma}\\
            1 - \frac{v}{c} &= \sqrt{1 - \frac{v^2}{c^2}}\\
            1 - 2\frac{v}{c} + \frac{v^2}{c^2} &= 1 - \frac{v^2}{c^2}\\
            2\frac{v^2}{c^2} - 2\frac{v}{c} &= 0\\
            \frac{v}{c}(\frac{v}{c} - 1) &= 0
        \end{align*}
        The solutions to this equation are $v=0$ and $v=c$ then the only way
        for the bomb not to explode is that the train travels above the speed
        of light which is not possible, therefore the bomb explodes.
    \end{proof}
	\begin{proof}{\textbf{1.8}}
        From the stick frame the runner is going to be length contracted in the
        direction in which the stick is falling.
        
        And from the runner frame the stick is going to be length contracted in
        the direction which is running the runner, but in addition to that
        due to the loss of simultaneity the stick is not going to hit flat
        the floor, the rear part of the stick is going to hit first the floor
        from the runner frame but then a cusp is going to form since the rear
        part starts to bounce back but the other parts of the stick are still
        falling.
    \end{proof}
	\begin{proof}{\textbf{1.10}}
        From the cookie-cutter reference frame, the cookie dough is going to be
        length contracted in the direction that the conveyor belt is moving
        but when the cookie-cutter cuts the cookie the dough is slowed down, and
        therefore the length contraction is undone to the cookie, so the cookie
        is going to be stretched in the direction of the belt.
        
        From the dough frame, the cookie-cutter is length contracted but in
        this frame, a loss of simultaneity is happening too, so the rear part
        of the circular cookie cutter is cutting first and then the frontal
        part, therefore, giving an "elliptical" form to the cookie, stretched
        on the conveyor belt direction.  
    \end{proof}
\cleardoublepage
	\begin{proof}{\textbf{1.13}}
        If $q$ is moving with velocity $v$ then from $q$'s frame the movement
        is seen in the opposite direction in the wire, and this means that the
        protons are moving too, as seen from $q$ the distance between them is
        length contracted so the charge density in this case is given by
        $$\lambda_{protons} = \gamma \lambda_0$$

        In the electrons case since they are already moving with velocity $v_0$
        from $q$'s frame the velocity is given by
        $$v_0'= \frac{v_0 - v}{1 - v_0v/c^2}$$
        now we want to find how much the length between electrons is
        contracted due to this movement. We know that the proper length between
        the electrons is length contracted as seen from the wire due to the
        velocity at which they move, $v_0$, so
        $$L_{proper} = \gamma_{v_0} L_{0}$$
        and from there we can determine the proper charge density as
        $$\lambda_{proper} = \frac{-\lambda_0}{\gamma_{v_0}}$$
        now that we have the proper length we can calculate the length
        contraction between each electron as seen by $q$ and therefore the
        charge density (which is multiplied by the $\gamma_{v_0'}$ factor) 
        $$\lambda_{v_0'} = \gamma_{v_0'}\lambda_{proper} = 
        \gamma_{v_0'}\frac{-\lambda_0}{\gamma_{v_0}}$$
        Now we need to calculate $\gamma_{v_0'}$ where we define
        $\beta_{v_0}' = v_0'/c$ then
        \begin{align*}
            \gamma_{v_0'} &= \frac{1}{\sqrt{1 - \beta_{v_0}^{'2}}}\\
                &= \frac{1}{\sqrt{1 - (\frac{\beta_{v_0} - \beta}{1-\beta_{v_0}\beta})^2}}\\
                &= \frac{1-\beta_{v_0}\beta}{\sqrt{(1-\beta_{v_0}\beta)^2 - (\beta_{v_0} - \beta)^2}}\\
                &= \frac{1-\beta_{v_0}\beta}{\sqrt{
                    1 - 2\beta_{v_0}\beta + \beta_{v_0}^2\beta^2 - \beta_{v_0}^2 + 2\beta_{v_0}\beta - \beta^2 
                }}\\
                &= \frac{1-\beta_{v_0}\beta}{\sqrt{\beta_{v_0}^2\beta^2 - \beta_{v_0}^2 - \beta^2 + 1}}\\
                &= \frac{1-\beta_{v_0}\beta}{\sqrt{\beta_{v_0}^2(\beta^2 - 1) -\beta^2 + 1}}\\
                &= \frac{1-\beta_{v_0}\beta}{\sqrt{(\beta^2 - 1)(\beta_{v_0}^2 -1)}}\\
                &= \gamma_{v_0}\gamma(1-\beta_{v_0}\beta)
        \end{align*}
        Therefore, the charge density for the electrons is
        $$\lambda_{electrons} = \gamma_{v_0'}\frac{-\lambda_0}{\gamma_{v_0}} =
        -\gamma\lambda_0(1-\beta_{v_0}\beta)$$
        and the net charge density is given by
        \begin{align*}
            \lambda_{net} &= \lambda_{protons} + \lambda_{electrons}\\
                &= \gamma\lambda_0 - \gamma\lambda_0(1-\beta_{v_0}\beta)\\
                &= \gamma\lambda_0\beta_{v_0}\beta
        \end{align*}
        which is non-zero as we wanted, assuming $\beta_{v_0}$ is positive,
        depending on the sign of $\beta$ the $\lambda_{net}$ is going to be
        positive or negative, meaning that the charge $q$ is going to move
        towards or away of the wire.
    \end{proof}
    \begin{proof}{\textbf{1.15}}
        Since the light beam moves at $c/n$ with respect to the water and the water
        moves at $v$ with respect to the ground then the relativistic way of computing
        the velocities is as follows
        \begin{align*}
            V &= \frac{\frac{c}{n} + v}{1 + \frac{cv}{nc^2}}\\
              &= \frac{\frac{c}{n} + v}{1 + \frac{v}{nc}}
        \end{align*}
        Let us assume now that $v \ll c$. Let's start by multiplying both numerator
        and denominator by $1-\frac{v}{nc}$ then we have that
        \begin{align*}
            \frac{
                (\frac{c}{n} + v)(1-\frac{v}{nc})
                }{
                (1 + \frac{v}{nc})(1-\frac{v}{nc})
            } &= \frac{
                c(\frac{1}{n} - \frac{v}{n^2c} + \frac{v}{c} - \frac{v^2}{nc^2})
                }{
                (1 -\frac{v}{nc} + \frac{v}{nc} -\frac{v^2}{n^2c^2})
            }
            \\
            &= \frac{
                c(\frac{1}{n} - \frac{v}{n^2c} + \frac{v}{c} - \frac{v^2}{nc^2})
                }{
                (1 -\frac{v^2}{n^2c^2})
            }
        \end{align*}
        Since we know that $v \ll c$ then we remove the terms involving 
        $\frac{v^2}{c^2}$ because we can assume that $\frac{v^2}{c^2} \ll 1$
        and therefore the equation becomes
        \begin{align*}
            c(\frac{1}{n} - \frac{v}{n^2c} + \frac{v}{c}) &= \frac{c}{n} - \frac{v}{n^2} + v\\
                &= \frac{c}{n} + (1-\frac{1}{n^2})v
        \end{align*}
        Then $A = 1-\frac{1}{n^2}$.
    \end{proof}
\cleardoublepage
	\begin{proof}{\textbf{1.16}}
        From $C$'s frame the velocities of $A$ and $B$ are
        $$u_A = \frac{\frac{4c}{5} - v}{1 - \frac{4v}{5c}}$$
        and
        $$u_B = \frac{\frac{3c}{5} - v}{1 - \frac{3v}{5c}}$$
        where $v$ is the unknown velocity at which $C$ is travelling.
        Since we want that the velocities of $A$ and $B$ be the same from $C$'s frame,
        then must happen that $u_A = -u_B$ so
        \begin{align*}
            \frac{\frac{4c}{5} - v}{1 - \frac{4v}{5c}} &= -\frac{\frac{3c}{5} - v}{1 - \frac{3v}{5c}} \\
            (\frac{4c}{5}-v)(1-\frac{3v}{5c}) &= (v - \frac{3c}{5})(1 - \frac{4v}{5c}) \\
            \frac{4c}{5} - \frac{12}{25}v - v + \frac{3v^2}{5c} &= v - \frac{4v^2}{5c} - \frac{3c}{5} + \frac{12}{25}v\\
            \frac{7c}{5}-\frac{24}{25}v - 2v + \frac{7v^2}{5c}
        \end{align*}
        Let us assume now that $c=1$ then we have that
        \begin{align*}
            \frac{7}{5}-\frac{74}{25}v + \frac{7v^2}{5} &= 0\\
            35-74v + 35v^2 &= 0
        \end{align*}
        The solutions to this equation are $v_1 = 5/7$ and $v_2 = 7/5$, since $v_2$
        would give us a value above $c$ then we keep $v_1$.
        
        Finally, by replacing $v_1$ value we get the velocity at which both $A$ and $B$
        will approach to $C$ as seen from $C$'s frame.
        \begin{align*}
            u &= \frac{\frac{4c}{5} -  \frac{5c}{7}}{1 - \frac{4}{5}\cdot \frac{5}{7}}\\
              &= \frac{\frac{3}{35}c}{\frac{15}{35}}\\
              &= \frac{1}{5}c
        \end{align*}
    \end{proof}
\cleardoublepage
	\begin{proof}{\textbf{1.18}}
        We want to solve this problem by mathematical induction so we start by proving
        the base step, we know that the velocity of the object seen from $S_1$ is
        $\beta_1$ so if we apply the given equation we have that
        \begin{align*}
            \beta_{(1)} &= \frac{P_{1}^{+} - P_{1}^{-}}{P_{1}^{+} + P_{1}^{-}} \\
                &= \frac{(1+\beta_1)-(1-\beta_1)}{(1+\beta_1)+(1-\beta_1)} \\
                &= \frac{2\beta_1}{2} = \beta_1
        \end{align*}
        Now suppose that $\beta_{(N-1)}$ holds, we want to prove that also holds for
        $\beta_{(N)}$. As we know, seen from $S_N$ the velocity $\beta_{(N)}$ of
        the object is the relativistic sum between $\beta_{(N-1)}$ (which is the 
        velocity of the object seen from $S_{N-1}$) and $\beta_N$ (The velocity at
        which $S_{N-1}$ is moving), then
        \begin{align*}
            \beta_{(N)} &= \frac{\beta_{(N-1)} + \beta_N}{1 + \beta_{(N-1)}\beta_{N}} \\
                &= \frac{
                    \frac{P_{N-1}^+ - P_{N-1}^-}{P_{N-1}^+ + P_{N-1}^-} + \beta_N
                    }{
                    1 + \frac{P_{N-1}^+ - P_{N-1}^-}{P_{N-1}^+ + P_{N-1}^-}\beta_{N}
                    } \\
                &= \frac{
                    \frac{P_{N-1}^+ - P_{N-1}^- + (P_{N-1}^+ + P_{N-1}^-)\beta_N}{P_{N-1}^+ + P_{N-1}^-}
                    }{
                    \frac{P_{N-1}^+ + P_{N-1}^- + (P_{N-1}^+ - P_{N-1}^-)\beta_N}{P_{N-1}^+ + P_{N-1}^-}
                    } \\
                &= \frac{
                    P_{N-1}^+(1+\beta_N) - P_{N-1}^-(1-\beta_N)
                }{
                    P_{N-1}^+(1+\beta_N) + P_{N-1}^-(1-\beta_N)
                }
        \end{align*}
        Here we used the assumption that the equation holds for $\beta_{(N-1)}$.
        Finally by the definition of $P_N^+$ and $P_N^-$ we have that
        $P_N^+ = P_{N-1}^+(1+\beta_N)$ and $P_N^- = P_{N-1}^-(1-\beta_N)$
        therefore
        \begin{align*}
            \beta_{(N)} &= \frac{P_{N}^+ - P_{N}^-}{P_{N}^+ + P_{N}^-}
        \end{align*}
    \end{proof}

	\begin{proof}{\textbf{1.24}}
        When Alice's watch shows a time $T$, she waves, but on Bob's watch a
        time $\gamma T$ has passed, at that moment he waves too.

        When Bob's waves at $\gamma T$ as seen from his watch, on Alice's watch
        the time shown is $\gamma^2 T$, that's when she waves.

        When Alice's waves at $\gamma^2 T$ as seen from her watch, on Bob's
        watch the time shown is $\gamma^3 T$, that's when he waves.
        And so on.
    \end{proof}
\cleardoublepage
	\begin{proof}{\textbf{1.26}}
        \begin{itemize}
            \item [(a)] On the ground frame the train is length contracted to
            $L/\gamma$ so for the back of the train to be at the end of the
            tunnel the train should pass the tunnel length, that from the
            ground is $L$, and the entire train should be outside the tunnel,
            then the entire length that should be covered by the train is
            $L + L/\gamma$, also we know that the train is travelling at speed
            $3/5c$, then
            $$t = \frac{L + L/\gamma}{3/5c}
                = \frac{L(1 + 1/\gamma)}{3/5c}
                = \frac{9/5L}{3/5c}
                = 3\frac{L}{c}$$
            where we used that $\gamma = \gamma_{3/5} = 5/4$
            \item [(b)] For the person to be at the end of the train at the
            moment the train leaves the tunnel, the person should cover the
            length of the tunnel $L$ in the time we calculated in (a), then
            $$v_{person} = \frac{L}{3\frac{L}{c}} = \frac{1}{3}c$$
            \item [(c)] The time it takes for the person to reach the back of
            the train at the end of the tunnel seen from the ground frame is
            $3L/c$ as we saw, then on the person's watch due to
            the time dilation the person sees
            $$t_{person} = \frac{t_{ground}}{\gamma}
                = \frac{\frac{3L}{c}}{\gamma}
                = \frac{\frac{3L}{c}}{\frac{3}{\sqrt{8}}}
                = \sqrt{8}\frac{L}{c}$$   
            where we used that $\gamma = \gamma_{1/3} = 3/\sqrt{8}$
            \end{itemize}
    \end{proof}
	\begin{proof}{\textbf{1.27}}
        \begin{itemize}
            \item [(a)] The horizontal span of the stick in $S'$ frame is given
            by $L\cos{30}$ but since $S'$ is moving with respect to the ground
            with a velocity of $4/5c$ this horizontal span is going to be length contracted seen from the
            ground, then
            $$H_{ground} = \frac{L \cos{30}}{\gamma_{4/5}}
                = \frac{L\sqrt{3}/2}{5/3}
                = L\frac{3\sqrt{3}}{10}$$
            where we used that $\gamma_{4/5} = 5/3$
            \item [(b)] The photons coming from the points that are furthest
            away from the observer are going to arrive late to the observer,
            in addition to that, since the stick is moving, the horizontal span
            of the stick should look longer. 
        \end{itemize}
    \end{proof}
	\begin{proof}{\textbf{1.32}}
    \begin{itemize}
        \item [(a)] From the pole's frame the length of the barn is $L/\gamma$
        because of the length contraction, so when the front part of the pole 
        reaches to the end of the barn the back part of the pole is still outside
        the barn then it has to travel a distance of $L - L/\gamma$ to reach
        the entrance of the barn which takes a time
        $$\Delta t = \frac{L - L/\gamma}{v}$$
        When the front part of the pole reaches the end of the barn,
        the watch at the start of the barn should read $-\frac{L - L/\gamma}{v}$
        so after a time $\frac{L - L/\gamma}{v}$ this watch reads $0$ as we
        wanted, therefore the watch at the end of the barn at that moment
        should read
        $$t = -\frac{Lv}{c^2} -\frac{L - L/\gamma}{v}$$

        \item [(b)] From the barn's frame which has a length of $L$ the pole is
        going to be length contracted to $L/\gamma$, this means that for the
        front of the pole to reach the end of the barn a distance of
        $L - L/\gamma$ should be covered, therefore the time it takes for the
        front of the pole to reach the end of the barn is
        $$\Delta t = \frac{L - L/\gamma}{v}$$
        in adition to that the watch on the front of the pole reads $-Lv/c^2$
        when the back of the pole match with the start of the barn so when the
        front of the pole matches with the end of the barn the front clock
        should read
        $$t = \frac{-Lv}{c^2} + \frac{L - L/\gamma}{v}$$
    \end{itemize}
    \end{proof}
	\begin{proof}{\textbf{1.35}}
    \begin{itemize}
        \item [(a)] From the ground frame due to the length contraction the
        train has a length of $L/\gamma$, and since the gap closing velocity
        is $c-v$ then the time it takes for the photon to reach the front of
        the train from the ground frame is $L/\gamma(c-v)$ so the distance
        $D$ the photon has to travel is given by
        \begin{align*}
            D &= \frac{cL}{\gamma(c-v)}
        \end{align*}
        \item [(b)] In this case to show that the time it takes for the front
        part of the train to reach the house is the same as $L/c$ (the time it
        takes for the photon to get to the front of the train), we length
        contract the result we got from part (a) because the train sees the
        distance between the tree and the house length contracted as
        $\frac{cL}{\gamma^2(c-v)}$ but this distance takes into account the
        length of the train so we subtract the length of the train to get the
        distance from the front of the train to the house as
        $$\frac{cL}{\gamma^2(c-v)} - L$$        
        Also, we know that the train travels at a velocity $v$ so we divide by
        $v$ to obtain the time it takes for the front part of the train to
        reach the house
        \begin{align*}
            \frac{\frac{cL}{\gamma^2(c-v)} - L}{v}
                &= \left[\frac{\frac{c(c^2-v^2)}{c^2(c-v)} - 1}{v}\right]L \\
                &= \left[\frac{(c^2-v^2)}{cv(c-v)} - \frac{1}{v}\right]L \\
                &= \left[\frac{c^2-v^2 - c(c-v)}{cv(c-v)}\right]L \\
                &= \left[\frac{- v^2 + cv}{cv(c-v)}\right]L \\
                &= \left[\frac{cv(1 - v/c)}{cv(c-v)}\right]L \\
                &= \left[\frac{1 - v/c}{c(1 - v/c)}\right]L \\
                &= \frac{L}{c}
        \end{align*}
        So after doing some algebra we got the result we wanted.
    \end{itemize}
    \end{proof}
\cleardoublepage
    \begin{proof}{\textbf{1.37}}
    \begin{itemize}
        \item [(a)] From the ground reference frame the person is going to
        travel a distance $fL$ before reaching the photon which is going to take
        a time $fL/v$. Then the photon is going to travel a distance $(1-f)L$
        before reaching the person and that is going to take a time $(1-f)L/c$.
        These times must be equal therefore
        \begin{align*}
            \frac{fL}{v} &= \frac{(1 - f)L}{c} \\
            f\frac{c}{v} &= 1 - f \\
            f\left(1 + \frac{c}{v}\right) &= 1 \\
            f &= \frac{1}{1 + \frac{c}{v}}
        \end{align*}
        \item [(b)] From the person's frame the tunnel is going to travel a
        distance $fL/\gamma$ before the person reaches the photon due to length
        contraction, this is going to take a time $fL/\gamma v$.
        On the other hand the photon is going to travel a distance
        $(1-f)L/\gamma$ before reaching the person which is going to take a
        time $(1-f)L/\gamma c$. These times must be equal therefore
        \begin{align*}
            \frac{fL}{\gamma v} &= \frac{(1 - f)L}{\gamma c} \\
            f\frac{c}{v} &= 1 - f \\
            f\left(1 + \frac{c}{v}\right) &= 1 \\
            f &= \frac{1}{1 + \frac{c}{v}}
        \end{align*}
    \end{itemize}        
    \end{proof}
    \begin{proof}{\textbf{1.40}}
        If $q$ is moving with velocity $v = 3c/5$ then from $q$'s frame the movement
        is seen in the opposite direction in the wire, and this means that the
        protons are moving too, as seen from $q$ the distance between them is
        length contracted so the charge density in this case is given by
        \begin{align*}
            \lambda_{protons} &= \gamma_{3/5} \lambda_0 \\
                &= \frac{5}{4} \lambda_0
        \end{align*}

        In the electrons case since they are already moving with velocity
        $v_0 = 4c/5$ from $q$'s frame the velocity is given by
        \begin{align*}
            v_0'&= \frac{4c/5 - 3c/5}{1 - 12c^2/25c^2}\\
                &= \frac{1/5}{13/25}c \\
                &= \frac{5}{13}c            
        \end{align*}
        now we want to find how much the length between electrons is
        contracted due to this movement. We know that the proper length between
        the electrons is length contracted as seen from the wire due to the
        velocity at which they move, $v_0 = 4c/5$, so
        \begin{align*}
            L_{proper} &= \gamma_{4/5} L_{0} \\
                &= \frac{5}{3} L_{0}
        \end{align*}
        and from there we can determine the proper charge density as
        $$\lambda_{proper}
            = \frac{-\lambda_0}{\gamma_{4/5}} 
            = -\frac{3}{5}\lambda_0$$
        now that we have the proper length we can calculate the length
        contraction between each electron as seen by $q$ and therefore the
        charge density for the electrons (which is multiplied by the
        $\gamma_{v_0'} = \gamma_{5/13}$ factor) 
        \begin{align*}
            \lambda_{v_0'}
                &= \gamma_{5/13}\lambda_{proper} \\
                &= \gamma_{5/13}\frac{-\lambda_0}{\gamma_{4/5}} \\
                &= -\left(\frac{13}{12}\right)\left(\frac{3}{5}\right)\lambda_0 \\
                &= -\frac{13}{20} \lambda_0
        \end{align*}
        Therefore the net charge density is given by
        \begin{align*}
            \lambda_{net} &= \lambda_{protons} + \lambda_{electrons}\\
                &= \frac{5}{4}\lambda_0 - \frac{13}{20}\lambda_0\\
                &= \frac{3}{5}\lambda_0
        \end{align*}
    \end{proof}
\cleardoublepage
    \begin{proof}{\textbf{1.41}}
        Let $\gamma_u = 1/\sqrt{1-u^2}$ and $\gamma_v = 1/\sqrt{1-v^2}$.
        Let us say that $w = (u + v)/(1 + uv)$ i.e. $w$ is the relativistic addition
        of $u$ and $v$, then the corresponding $\gamma$ to $w$ is given by  
        \begin{align*}
            \gamma &= \frac{1}{\sqrt{1-w^2}}\\
                &= \frac{1}{\sqrt{1-\frac{(u+v)^2}{(1+uv)^2}}}\\
                &= \frac{1}{\sqrt{\frac{(1+uv)^2 -(u+v)^2}{(1+uv)^2}}}\\
                &= \frac{1}{\frac{\sqrt{1+2uv+u^2v^2 - (u^2 + 2uv +v^2)}}{1+uv}}\\
                &= \frac{1+uv}{\sqrt{1+u^2v^2 -u^2 -v^2}}\\
                &= \frac{1+uv}{\sqrt{1-u^2 -v^2(1-u^2)}}\\
                &= \frac{1+uv}{\sqrt{(1-u^2)(1-v^2)}} = \gamma_u \gamma_v (1+uv)\\
        \end{align*}
        In case $w = (u - v)/(1 - uv)$ i.e. $w$ is the relativistic subtraction
        between $u$ and $v$ we have that
        \begin{align*}
            \gamma &= \frac{1}{\sqrt{1-w^2}}\\
                &= \frac{1}{\sqrt{1-\frac{(u-v)^2}{(1-uv)^2}}}\\
                &= \frac{1}{\sqrt{\frac{(1-uv)^2 -(u-v)^2}{(1-uv)^2}}}\\
                &= \frac{1}{\frac{\sqrt{1-2uv+u^2v^2 - (u^2 - 2uv +v^2)}}{1-uv}}\\
                &= \frac{1-uv}{\sqrt{1+u^2v^2 -u^2 -v^2}}\\
                &= \frac{1-uv}{\sqrt{1-u^2 -v^2(1-u^2)}}\\
                &= \frac{1-uv}{\sqrt{(1-u^2)(1-v^2)}} = \gamma_u \gamma_v (1-uv)\\
        \end{align*}
    \end{proof}
\cleardoublepage
    \begin{proof}{\textbf{1.44}}
        Let us determine first the time it takes for the ball to travel to the front of
        the train from the ground frame. From the ground frame the length of the
        train is $L/\gamma$ and the ball is travelling at a velocity given by the 
        relativistic subtraction of the velocities of the ball and the train i.e
        $\frac{u-v}{1-uv/c^2}$ so the time it takes for the ball to get to the front of
        the train from the ground frame is
        \begin{align*}
            t &= \frac{L}{\gamma(\frac{u-v}{1-uv/c^2})} \\
              &= \frac{L(1-uv/c^2)}{\gamma(u-v)} 
        \end{align*}
        On the other hand, in the train frame the time it takes for the ball to go from
        the back of the train to the front is $L/u$. Now if we suppose there is a clock
        on the back of the train which reads $0$ at the moment the front of the
        train passes the tree seen from the ground frame then this clock should read
        $Lv/c^2 + L/u$ when the back of the train passes the tree due to the rear clock
        ahead effect. Then the time elapsed in the ground frame taking into account the
        time dilation effect is
        \begin{align*}
            t = \gamma(\frac{Lv}{c^2} + \frac{L}{u})
        \end{align*}
        Therefore if we use both equations we get that
        \begin{align*}
            \gamma(\frac{Lv}{c^2} + \frac{L}{u}) &= \frac{L(1-uv/c^2)}{\gamma(u-v)}\\
            \gamma^2(u-v)(\frac{v}{c^2} + \frac{1}{u}) &= \frac{c^2-uv}{c^2}\\
            \gamma^2(u-v)(\frac{uv+c^2}{c^2u}) &= \frac{c^2-uv}{c^2}\\
            \gamma^2(u-v)(uv+c^2) &= uc^2-u^2v\\
            \gamma^2(u^2v + uc^2-uv^2-vc^2) &= uc^2-u^2v\\
            u^2(v(\gamma^2+1)) + u(c^2(\gamma^2-1) - v^2) - vc^2 = 0
        \end{align*}
        Solving this equation we get the value of $u$ so that the ball
        hits the front simultaneously (as measured in the ground frame) at the moment
        the back of the train is passing the tree. The solution is given by
        \begin{align*}
            u = \frac{
                -(c^2(\gamma^2-1) - v^2) \pm \sqrt{(
                    c^2(\gamma^2-1) - v^2)^2 + 4v^2c^2(\gamma^2+1)
                }
            }{2v(\gamma^2+1)}
        \end{align*}
        Now the valid solutions to this equation must be real therefore the discriminant
        must be bigger or equal to 0. Also, let's recall that
        $$\gamma^2 - 1 = \frac{c^2}{c^2-v^2} -1 = \frac{v^2}{c^2-v^2}$$
        and that 
        $$\gamma^2 + 1 = \frac{c^2}{c^2-v^2} +1 = \frac{2c^2 - v^2}{c^2-v^2}$$
        then
        \begin{align*}
            (c^2(\gamma^2-1) - v^2)^2 + 4v^2c^2(\gamma^2+1) &=
                (\frac{c^2v^2}{c^2-v^2} - v^2)^2 + 4v^2c^2(\frac{2c^2 - v^2}{c^2-v^2})\\
                &= v^4(\frac{c^2}{c^2-v^2} - 1)^2 + 4v^2c^2(\frac{2c^2 - v^2}{c^2-v^2})\\
                &= \frac{v^8}{(c^2-v^2)^2} + 4v^2c^2(\frac{2c^2 - v^2}{c^2-v^2})\\
                &= \frac{v^8 + 4v^2c^2(2c^2 - v^2)(c^2-v^2)}{(c^2-v^2)^2}
        \end{align*}
        Then if $v<c$ as we know it should be, then the equation is positive.
    \end{proof}
    \begin{proof}{\textbf{1.49}}
        The time it takes for the bullet to go from the back to the front of the train
        in the train frame is $L/u$ and this elapsed time in the ground frame due to the
        time dilation effect becomes $\gamma(L/u)$.
        
        On the other hand, the velocity seen from the ground frame at which the bullet
        is travelling is $\frac{u-v}{1-(uv/c^2)}$. So we have that
        $$\frac{u-v}{1-(uv/c^2)} = \frac{x}{\gamma(L/u)}$$
        where $x$ is the distance travelled in the lapse $\gamma(L/u)$ seen from the
        ground frame, therefore
        $$x = L\left(\frac{\gamma(u-v)}{u-(u^2v/c^2)}\right)$$
        When the first bullet is at $x$ the next bullet is fired, also given that the
        train is length contracted then it measures $L/\gamma$ seen from the ground
        frame, so the number of bullets in flight simultaneously (from the ground frame)
        is 
        \begin{align*}
            L/\gamma &= nx\\
            1/\gamma &= n\left(\frac{\gamma(u-v)}{u-(u^2v/c^2)}\right)\\
            n &= \frac{u-(u^2v/c^2)}{\gamma^2(u-v)}             
        \end{align*}
    \end{proof}

\end{document}























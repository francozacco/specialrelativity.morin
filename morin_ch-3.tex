\documentclass[11pt]{article}
\usepackage{amssymb}
\usepackage{amsthm}
\usepackage{enumitem}
\usepackage{amsmath}
\usepackage{bm}
\usepackage{adjustbox}
\usepackage{mathrsfs}
\usepackage{graphicx}
\usepackage{siunitx}
\usepackage[mathscr]{euscript}

\title{\textbf{Solved selected problems of Special Relativity - Morin}}
\author{Franco Zacco}
\date{}

\addtolength{\topmargin}{-3cm}
\addtolength{\textheight}{3cm}

\newcommand{\hatr}{\bm{\hat{r}}}
\newcommand{\hatx}{\bm{\hat{x}}}
\newcommand{\haty}{\bm{\hat{y}}}
\newcommand{\hatz}{\bm{\hat{z}}}
\newcommand{\hatth}{\bm{\hat{\theta}}}
\newcommand{\hatphi}{\bm{\hat{\phi}}}
\newcommand{\hatrho}{\bm{\hat{\rho}}}
\theoremstyle{definition}
\newtheorem*{solution*}{Solution}
\renewcommand*{\proofname}{Solution}

\begin{document}
\maketitle
\thispagestyle{empty}

\section*{Chapter 3 - Dynamics}

\begin{proof}{\textbf{3.1}}
    Let the mass in the rest frame have the initial energy $E_0$ then by
    conservation of energy each photon must have $E_0/2 = hf/2$ energy
    after the decay.
    
    Then in the frame where the mass is moving with velocity $v$, we have
    to take into account the relativistic Doppler effect hence the energy
    of the photons is given by
    \begin{align*}
        \frac{hf}{2}\sqrt{\frac{1+ v}{1-v}}
        \qquad\text{and}\qquad
        \frac{hf}{2}\sqrt{\frac{1 -v}{1 +v}}
    \end{align*}
    Where we assumed that each photon goes in a different direction.
    Then the total energy of the photons is
    \begin{align*}
        E &= \frac{E_0}{2}\sqrt{\frac{1+ v}{1-v}}
        + \frac{E_0}{2}\sqrt{\frac{1 -v}{1 +v}}\\
        &= \frac{E_0}{2}\bigg((1+v)\sqrt{\frac{1}{1-v^2}} 
        + (1-v)\sqrt{\frac{1}{1 -v^2}}\bigg)\\
        &= \frac{E_0}{\sqrt{1-v^2}} = \gamma E_0
    \end{align*}
    Also since $E_0$ is the energy of the rest mass and $\gamma E_0$ is
    the energy of a moving mass with velocity $v$ then $\gamma E_0 - E_0$
    must reduce to $mv^2/2$ in the non-relativistic limit then
    \begin{align*}
        \frac{E_0}{\sqrt{1-\frac{v^2}{c^2}}} - E_0 &= \frac{mv^2}{2}\\
        E_0\bigg(\bigg(1 + \frac{(v/c)^2}{2}\bigg) - 1\bigg) &= \frac{mv^2}{2}\\
        E_0\frac{(v/c)^2}{2} &= \frac{mv^2}{2}\\
        E_0 &= mc^2
    \end{align*}
    Where we approximated by Taylor series $1/\sqrt{1 - (v/c)^2}$ as 
    $1 + (v/c)^2/2 + O((v/c)^4)$. Therefore the total energy is given by
    $E = \gamma mc^2$.

    On the other hand, let $p_0$ be the momentum of the mass in the frame
    where it is traveling with velocity $v$ then when the decay happens
    the momentum of the photon must take into account the relativistic
    Doppler effect and the direction of each photon so we will have 
    opposite signs for each momentum, hence the total momentum will be
    \begin{align*}
        p &= \frac{p_0}{2}\sqrt{\frac{1+ v}{1-v}}
        - \frac{p_0}{2}\sqrt{\frac{1 - v}{1 + v}}\\
            &= \frac{p_0}{2}\bigg(\frac{1}
            {\sqrt{1 - v^2}}((1 + v) - (1 - v))\bigg)\\
            &= \gamma v p_0
    \end{align*}
    By adding the missing $c$ we have that $p = \gamma(v/c)p_0$.
    Since $p$ must reduce to $mv$ in the non-relativistic limit where
    $\gamma \approx 1$ we must have that $p_0 = mc$. Therefore the total
    momentum is given by $p = \gamma mv$.
\end{proof}
\begin{proof}{\textbf{3.2}}
    Let the particle be traveling at $u'$ velocity with respect to the $S'$
    reference frame with an energy $E' = E_{total}$ and a momentum
    $p' = p_{total}$. Since the velocity of the particle with respect to the
    CM reference frame is 0 then the CM reference frame $S_{CM}$ must be
    traveling at the particle velocity $u$ but then seen from the CM reference
    frame the $S'$ reference frame is moving with a velocity $-u$.    
    Then from the (Energy/Momentum) Lorentz transformation we have that
    \begin{align*}
        E_{CM} &= \gamma_u(E' - up')\\
        0 &= \gamma_u(p' - uE'/c^2)
    \end{align*}
    Therefore from the last equation, we have that the velocity of the frame in
    which the total momentum is zero is
    $$u = c^2~\frac{p_{total}}{E_{total}}$$. 
\end{proof}
\cleardoublepage
\begin{proof}{\textbf{3.3}}
    Let the mass resulting from the collision be $M'$ then from the
    conservation of energy and momentum we have that
    \begin{align*}
        E_{M'} &= E_{M} + E_{m} = \gamma M + m\\
        p_{M'} &= p_{M} + p_{m} = \gamma M v + 0
    \end{align*}
    Where we dropped the $c$'s. Also, we know that $M'^2 = E_{M'}^2 - p_{M'}^2$
    hence
    \begin{align*}
        M' &= \sqrt{(\gamma M + m)^2 - (\gamma M v)^2}\\
           &= \sqrt{\gamma^2 M^2 + 2\gamma M m +m^2 - \gamma^2 M^2 v^2}\\
           &= \sqrt{\gamma^2 M^2(1 - v^2) + 2\gamma M m + m^2}\\
           &= \sqrt{M^2 + 2\gamma M m + m^2}
    \end{align*}
    Here we used that $\gamma^2 (1 - v^2) = 1$. Now if we consider that
    $m \ll M$ we can drop the term $m^2$ hence
    \begin{align*}
        M' &\approx \sqrt{M^2 + 2\gamma M m}\\
           &\approx M\sqrt{1 + \frac{2\gamma m}{M}}\\
           &\approx M + \gamma m
    \end{align*}
    We went from the second to the last step by applying a binomial
    approximation which we can do because $2\gamma m/M < 1$.
\end{proof}
\cleardoublepage
\begin{proof}{\textbf{3.4}}
    Let us assume, without loss of generality, that after the decay the
    particles $M_B$ and $M_C$ have only a momentum $p_B$ and $p_C$ respectively
    in the $x$ direction.

    Let's write the 4-momenta before the decay for the particle $M_A$
    (dropping the $c$'s) where we have that
    \begin{align*}
        P = (M_A, 0, 0, 0)
    \end{align*}
    After the decay, we will have that
    \begin{align*}
        P_B = (\gamma M_B, p_B, 0, 0)\qquad P_C = (\gamma M_C, p_C, 0, 0)        
    \end{align*}
    Then by the conservation laws of energy and momentum, we get that
    \begin{align*}
        M_A &= \gamma M_B + \gamma M_C\\
        p_B &= -p_C
    \end{align*}
    We see that both particles have opposite velocity directions.

    On the other hand, we know that
    \begin{align*}
        p_B^2 = E_B^2 - M_B^2 = E_C^2 - M_C^2 = p_C^2
    \end{align*}
    Hence
    \begin{align*}
        E_B^2 = E_C^2 - M_C^2 + M_B^2\qquad E_C^2 = E_B^2 - M_B^2 + M_C^2
    \end{align*}
    So if we replace $E_C = M_A - \gamma M_B$ and knowing that
    $E_B = \gamma M_B$ we get that
    \begin{align*}
        E_B^2 &= (M_A - \gamma M_B)^2 - M_C^2 + M_B^2\\
        E_B^2 &= M_A^2 - 2\gamma M_BM_A + (\gamma M_B)^2 - M_C^2 + M_B^2\\
        0 &= M_A^2 - 2E_BM_A - M_C^2 + M_B^2\\
        E_B &= \frac{M_A^2 + M_B^2 - M_C^2}{2M_A}
    \end{align*}
    In the same way, replacing $E_B = M_A - \gamma M_C$ in the other equation
    and knowing that $E_C = \gamma M_C$ we get that
    \begin{align*}
        E_C^2 &= (M_A - \gamma M_C)^2 - M_B^2 + M_C^2\\
        % E_C^2 &= M_A^2 - 2\gamma M_CM_A + (\gamma M_C)^2 - M_B^2 + M_C^2\\
        % 0 &= M_A^2 - 2E_CM_A - M_B^2 + M_C^2\\
        E_C &= \frac{M_A^2 + M_C^2 - M_B^2}{2M_A}
    \end{align*}
    Finally, we can determine the momentum of particles $M_C$ and $M_B$ as
    follows
    \begin{align*}
        p_B = -p_C &= \sqrt{
            \bigg(\frac{M_A^2 + M_B^2 - M_C^2}{2M_A}\bigg)^2
            - M_B^2
        }\\
        &= \sqrt{
            \frac{(M_A^2 + M_B^2 - M_C^2)^2 - 4M_A^2M_B^2}{4M_A^2}
        }\\
        &= \frac{1}{2M_A} \sqrt{(M_A^2 + M_B^2 - M_C^2)^2 - 4M_A^2M_B^2}
    \end{align*}

\end{proof}
\cleardoublepage
\begin{proof}{\textbf{3.5}}
    Let a particle with mass $m$ and energy $E$ collide with an identical
    stationary particle.

    From the conservation energy equation, where we dropped the $c$'s,
    we want to produce $N$ particles that will have $E'$ energy, hence
    \begin{align*}
        E + m = NE'
    \end{align*}
    Analogously from the momentum conservation equation, we would want that
    \begin{align*}
        p = Np'
    \end{align*}
    where $p'$ is the momentum of each produced particle. Also, we know that
    $E'^2 = m^2 + p'^2 $ and $p^2 = E^2 - m^2$ then by squaring the energy
    conservation equation and replacing the values of $E'^2$ and $p^2$ we get
    that
    \begin{align*}
        (E + m)^2 &= (NE')^2\\
        (E + m)^2 &= N^2(m^2 + p'^2)\\
        (E + m)^2 &= N^2m^2 + p^2\\
        E^2 + 2Em + m^2 &= N^2m^2 + E^2 - m^2\\
        2Em &= N^2m^2 - 2m^2\\
        E &= m\bigg(\frac{N^2}{2} - 1\bigg)
    \end{align*}
    Which gives us the minimum energy that we would need to produce $N$
    particles with $m$ (rest) mass.
\end{proof}
\cleardoublepage
\begin{proof}{\textbf{3.6}}
    Let a particle with mass $M$ decay into several particles where one of
    them has a mass of $m$ and the sum of the rest sum up to a mass of $\mu$.

    From the energy conservation equation, where we dropped the $c$'s we have
    that
    \begin{align*}
        M &= E_m + E_\mu\\
        M - E_m &= E_\mu
    \end{align*}
    Also, from the momentum conservation equation, we see that
    \begin{align*}
        0 &= p_m + p_\mu\\
        -p_m &= p_\mu
    \end{align*}
    By squaring both sides of the energy conservation equation we get that
    \begin{align*}
        (M - E_m)^2 &= E_\mu^2\\
        M^2 - 2ME_m + E_m^2 &= E_\mu^2
    \end{align*}
    Also we know that $E_\mu^2 -p_\mu^2 = E_{CM}^2$ where $E_{CM}^2$ is the
    total energy in the center-of-mass frame for the $\mu$ particles and that
    $p_m^2 = p_\mu^2$ hence
    \begin{align*}
        M^2 - 2ME_m + E_m^2 -p_\mu^2 &= E_\mu^2 -p_\mu^2\\
        M^2 - 2ME_m + E_m^2 -p_m^2 &= E_{CM}^2
    \end{align*}
    So by replacing $m^2 = E_m^2 - p_m^2$ we get that
    \begin{align*}
        M^2 - 2ME_m + m^2 &= E_{CM}^2\\
        M^2 - 2ME_m  &= E_{CM}^2  - m^2\\
        2ME_m  &= m^2 + M^2 - E_{CM}^2\\
        E_m &= \frac{m^2 + M^2 - E_{CM}^2}{2M}
    \end{align*}
    We can maximize $E_m$ by minimizing $E_{CM}$ which is going to happen
    when all of the $\mu$ particles are at rest in the CM frame hence
    $E_{CM} = \mu$. Therefore $E_m$ is going to be the maximum when
    \begin{align*}
        E_m &= \frac{m^2 + M^2 - \mu^2}{2M}        
    \end{align*}
\end{proof}
\cleardoublepage
\begin{proof}{\textbf{3.10}}
    Before the collision, we see that
    \begin{align*}
        P_p &=\bigg(\frac{hc}{\lambda}, ~\frac{hc}{\lambda},~ 0,~0\bigg)\\
        P_m &= \bigg(mc^2,~0,~0,~0\bigg)
    \end{align*}
    Where $P_p$ is the 4-momentum for the photon and $P_m$ is the 4-momentum
    for the electron.
    Then after the collision, we have that
    \begin{align*}
        P_p' &=\bigg(\frac{hc}{\lambda'}, ~\frac{hc}{\lambda'}\cos\theta,
        ~\frac{hc}{\lambda'}\sin\theta,~0\bigg)\\
        P_m' &= \bigg(E',~p_x'c,~p_y'c,~0\bigg)
    \end{align*}
    From the conservation of energy and momentum, we know that
    $P_p + P_m = P_p' + P_m'$ and hence
    \begin{align*}
        P_m'^2 &= (P_p + P_m - P_p')^2\\
        P_m'^2 &= P_p^2 + 2P_p(P_m - P_p') + (P_m - P_p')^2\\
        P_m'^2 &= P_p^2 + 2P_pP_m - 2P_pP_p' + P_m^2 - 2P_mP_p' + P_p'^2\\
        m^2c^4 &= 0 + 2\frac{hc}{\lambda}mc^2 - 2\bigg(\frac{h^2c^2}{\lambda\lambda'}
        - \frac{h^2c^2}{\lambda\lambda'}\cos\theta\bigg) + m^2c^4
        - 2\frac{hc}{\lambda'}mc^2 + 0\\
        0 &= \frac{hc}{\lambda}mc^2 - \frac{h^2c^2}{\lambda\lambda'}
        + \frac{h^2c^2}{\lambda\lambda'}\cos\theta
        - \frac{hc}{\lambda'}mc^2 \\
        0 &= hmc^3\lambda' - h^2c^2 + h^2c^2\cos\theta - hmc^3\lambda\\
        \lambda' &= \lambda + \frac{h}{mc} - \frac{h}{mc}\cos\theta
    \end{align*}
    Therefore the wavelength $\lambda'$ in terms of $\lambda$ is given by
    \begin{align*}
        \lambda' &= \lambda + \frac{h}{mc}(1 - \cos\theta)
    \end{align*}
\end{proof}
\cleardoublepage
\begin{proof}{\textbf{3.11}}
    Let us compute the increase in speed in $S$ i.e. $a_x dt$ by adding
    relativistically  $a_x'dt'$ to a given speed $v$ hence
    \begin{align*}
        v + a_x dt = \frac{a_x' dt' + v}{1 + a_x' dt' v}
    \end{align*}
    also by time dilation, we know that the observed time in $S$ is slowed down
    by a factor $\gamma$ hence $dt = \gamma dt'$ so by replacing we have that
    \begin{align*}
        \frac{a_x' dt' + v}{1 + a_x' dt' v} &= v + a_x \gamma dt'\\
        a_x' dt' + v &= (v + a_x \gamma dt')(1 + a_x' dt' v)\\
        a_x' dt' + v &=
        v + a_x \gamma dt' + a_x' dt' v^2 + \gamma a_x a_x' dt'^2 v\\
        a_x' dt' &= a_x \gamma dt' + a_x' dt' v^2\\
        a_x' (1 - v^2) &= a_x \gamma\\
        a_x' &= a_x \gamma^3
    \end{align*}
    Where we dropped terms of order $dt'^2$ and we used that
    $\gamma^2 = 1/(1-v^2)$.
\end{proof}
\cleardoublepage
\begin{proof}{\textbf{3.12}}
    Let us take a small period of time $\Delta t$ where the velocity of the
    stick goes from $0$ to $\Delta u$ because of the force applied, then
    each dumbbell's mass will have a velocity with respect to
    the stick which can be determined by relativistically adding/subtracting
    $v$ and $\Delta u$ since we can assume that the time is so small that the
    masses still move horizontally hence
    \begin{align*}
        u = \frac{v \pm \Delta u}{1 \pm v\Delta u}
    \end{align*}
    This implies that the momentum for the mass moving in the positive
    direction ($\Delta u$ direction) is given by 
    \begin{align*}
        p_u &= \gamma_u m u\\
        &= \gamma_{\Delta u} \gamma_v (1 + v\Delta u)m
        \bigg(\frac{v + \Delta u}{1 + v\Delta u}\bigg)\\
        &= \gamma_{\Delta u} \gamma_v (v+ \Delta u)m
    \end{align*}
    Then the momentum for the mass moving in the negative direction is
    given by $-\gamma_{\Delta u} \gamma_v (v - \Delta u)m$.
    Where we used that $\gamma_u = \gamma_{\Delta u} \gamma_v (1 \pm v\Delta u)$.
    So the change in momentum of the system after a period $\Delta t$ is
    \begin{align*}
        \Delta p &= \gamma_{\Delta u} \gamma_v (v + \Delta u)m
        - \gamma_{\Delta u} \gamma_v (v - \Delta u)m - 0\\
        &= 2 \gamma_{\Delta u} \gamma_v m \Delta u
    \end{align*}
    but since $\Delta u$ is really small then $\gamma_{\Delta u} \approx 1$
    hence $p = 2\gamma_v m \Delta u$. Also we have that $F = \Delta p / \Delta t$
    so we get that 
    \begin{align*}
        F = \frac{\Delta p}{\Delta t} &= 2\gamma_v m \frac{\Delta u}{\Delta t}\\
        &= 2\gamma m a
    \end{align*}
    Where we replaced $a = \Delta u/\Delta t$ as the acceleration we added to
    the system and $\gamma_v = \gamma$ which implies that the system behaves
    like a mass $M = 2\gamma m$.    
\end{proof}
\cleardoublepage
\begin{proof}{\textbf{3.13}}
    We know from the relativistic Newton's equation that
    $$F = \gamma^3ma$$
    but also we know that $F$ is equal to Hooke's spring force so
    we have that
    \begin{align*}
        \gamma^3ma &= -m\omega^2x\\
        \gamma^3v\frac{dv}{dx} &= -\omega^2x
    \end{align*}
    where we used that $a = vdv/dx$, now by integrating we get that
    \begin{align*}
        \int \gamma^3vdv &= \int -\omega^2xdx\\
        \frac{1}{\sqrt{1-v^2}} &= -\omega^2 \frac{x^2}{2} + C\\
        \gamma &= -\frac{\omega^2x^2}{2} + C
    \end{align*}
    From the initial conditions, when $x = b$ we have that $v = 0$ then
    $C = 1 + \frac{\omega^2b^2}{2}$ so replacing and adding the $c$'s to fit
    the units we have that
    \begin{align*}
        \gamma &= 1 + \frac{\omega^2}{2c^2}(b^2 - x^2)
    \end{align*}
    Now we need to solve the differential equation we have to determine the
    period. Here we named $A = \frac{\omega^2}{2c^2}$ for simplicity hence 
    \begin{align*}
        1 + A(b^2 - x^2) &= \frac{1}{\sqrt{1 - (\frac{dx}{dt}/c)^2}}\\
        1 - \bigg(\frac{dx}{dt}\bigg)^2\frac{1}{c^2} &= \frac{1}{(1 + A(b^2 - x^2))^2} \\
        \bigg(\frac{dx}{dt}\bigg)^2\frac{1}{c^2} &= 1 - \frac{1}{(1 + A(b^2 - x^2))^2} \\
        \frac{dx}{dt} &=
        \frac{c\sqrt{(1 + A(b^2 - x^2))^2 - 1}}{1 + A(b^2 - x^2)}
    \end{align*}
    Since we know $\gamma = 1 + A(b^2 - x^2)$ by replacing and solving the
    differential equation we get that
    \begin{align*}
        \int_0^{T/4} dt &= \int_0^b \frac{\gamma}{c\sqrt{\gamma^2 - 1}} dx\\
        \frac{T}{4} &= \frac{1}{c}\int_0^b \frac{\gamma}{\sqrt{\gamma^2 - 1}} dx\\
        T &= \frac{4}{c}\int_0^b \frac{\gamma}{\sqrt{\gamma^2 - 1}} dx
    \end{align*}
\end{proof}
\cleardoublepage
\begin{proof}{\textbf{3.15}}
    Let the dustpan have a mass $M$ after some time $t$ in the dustpan frame.
    Then after $dt$ time the energy would change from $\gamma M$ to
    $\gamma M + \lambda dx = \gamma M + \lambda v dt$ and the momentum would be
    $\gamma M v$ also, we know that $M^2 = E^2 - p^2$ hence
    \begin{align*}
        (M + dM)^2 &= (\gamma M + \lambda v dt)^2 - (\gamma M v)^2\\
        M + dM &= \sqrt{(\gamma M)^2 + 2\gamma M\lambda v dt + (\lambda v dt)^2 - (\gamma M v)^2}\\
        M + dM &\approx \sqrt{(\gamma M)^2 + 2\gamma M\lambda v dt - (\gamma M v)^2}\\
        M + dM &\approx \sqrt{2\gamma M\lambda v dt + (\gamma M)^2(1 - v^2)}\\
        M + dM &\approx M\sqrt{\frac{2\gamma \lambda v dt}{M} + 1}
    \end{align*}
    Where we dismissed the term involving $dt^2$. On the other hand, by
    applying the binomial approximation we have that
    \begin{align*}
        M + dM &\approx M\bigg(1 + \frac{\gamma \lambda v dt}{M}\bigg)\\
        dM &\approx \gamma \lambda v dt
    \end{align*}
    Therefore the rate at which the mass of the dustpan plus the dust inside
    is increasing is
    $$\frac{dM}{dt} \approx \gamma \lambda v$$
\end{proof}
\cleardoublepage
\begin{proof}{\textbf{3.16}}
    The energy of the dustpan at the beginning is $\gamma_0 M_0$ in the 
    lab frame hence after a distance of $x$ the energy will be
    $\gamma M = \gamma_0 M_0 + \lambda x$. Also, we know that the momentum is
    conserved then at any moment the momentum must be $\gamma_0M_0v_0$
    hence when a distance $x$ has been covered, we can write
    that $\gamma Mv = \gamma_0M_0v_0$ so we have that
    \begin{align*}
        \gamma_0 M_0 + \lambda x &= \frac{\gamma_0 M_0 v_0}{v}\\
        v &= \frac{\gamma_0 M_0 v_0}{\gamma_0 M_0 + \lambda x}
    \end{align*}
    Which is the function $v(x)$ we are looking for.

    Now from the $v(x)$ equation, we can determine $x(t)$ by integration as follows
    \begin{align*}
        \frac{dx}{dt} &= \frac{\gamma_0 M_0 v_0}{\gamma_0 M_0 + \lambda x}\\
        \int(\gamma_0 M_0 + \lambda x)dx &= \int\gamma_0 M_0 v_0 dt\\
        \gamma_0 M_0x + \frac{\lambda x^2}{2} &= \gamma_0 M_0 v_0 t + C\\
        \frac{\lambda x^2}{2\gamma_0 M_0} + x &= v_0 t
    \end{align*}
    Where we used that $x=0$ when $t=0$ hence the constant of integration $C =0$.
    By solving for $x$ we get that
    \begin{align*}
        x &= \frac{-1\pm \sqrt{1
        + \frac{4\lambda v_0t}{2\gamma_0 M_0}}}{\frac{2\lambda}{2\gamma_0 M_0}}\\
        x &= \frac{-\gamma_0 M_0
        \pm \gamma_0 M_0\sqrt{1 + \frac{2\lambda v_0t}{\gamma_0 M_0}}}{\lambda}\\
        x &= \frac{-\gamma_0 M_0 \pm
        \sqrt{(\gamma_0 M_0)^2 + 2\lambda v_0\gamma_0 M_0 t}}{\lambda}
    \end{align*}
    Finally, to obtain $v(t) = dx/dt$ we derivate the above equation with
    respect to $t$ as follows
    \begin{align*}
        v(t) &=\frac{2\lambda v_0\gamma_0 M_0}
        {2\lambda\sqrt{(\gamma_0 M_0)^2 + 2\lambda v_0\gamma_0 M_0 t}}\\
        v(t) &=\frac{ v_0}
        {\sqrt{1 + \frac{2\lambda v_0 t}{\gamma_0 M_0}}}
    \end{align*}
\end{proof}
\cleardoublepage
\begin{proof}{\textbf{3.17}}
    From the dustpan frame, the dust is traveling with a velocity $v$ towards
    the dustpan, if we analyze a small timeframe $dt'$ in the dustpan frame
    we see that the dust mass colliding with the dustpan must be
    $\gamma\lambda vdt'$ because of length contraction. So the dustpan momentum 
    is going to change by $-\gamma(\gamma\lambda vdt')v$ were we
    used a negative sign since the dust is traveling toward the dustpan hence
    this is the change in momentum $dp = -\gamma^{2}\lambda v^2dt'$ which
    implies that the force exerted on the dustpan is
    \begin{align*}
        F = \frac{dp}{dt'} = -\gamma^{2}\lambda v^2
    \end{align*}
    If we work now from the lab frame we see that the dustpan is traveling
    towards the dust with a velocity $v$ so after a time $dt$ in the lab frame
    the dustpan must have collided (and grabbed) a dust mass $dm$ of
    $\gamma \lambda dx = \gamma \lambda vdt$ because of what we computed
    in problem 3.15 which changes the momentum of the dustpan by
    $dp = -\gamma (\gamma\lambda vdt) v$. 
    Therefore the force exerted on the dustpan is
    \begin{align*}
        F = \frac{dp}{dt} = -\gamma^{2}\lambda v^2
    \end{align*}
    Which has the same magnitude as the one we computed from the dustpan frame. 
\end{proof}

\cleardoublepage
\begin{proof}{\textbf{3.31}}
    The 4-momentum vector before the decay (where we dropped the c's) is given
    by
    \begin{align*}
        P = (\gamma m, \gamma mv, 0, 0)
    \end{align*}
    After the decay, the photons will have the following 4-momentum vectors
    \begin{align*}
        P_{ph1} &= (E_1, E_1, 0, 0)\\
        P_{ph2} &= \bigg(E_2, -\frac{E_2}{2}, \frac{\sqrt{3}E_2}{2}, 0\bigg)\\
        P_{ph3} &= \bigg(E_3, -\frac{E_3}{2}, -\frac{\sqrt{3}E_3}{2}, 0\bigg)
    \end{align*}
    Where we are considering the $x$ positive direction to the right and the
    $y$ positive direction upwards. Also, we used that $\sin 30 = 1/2$ and
    $\cos 30 = \sqrt{3}/2$.

    From the conservation of energy and momentum we have that
    \begin{align}
        \gamma m &= E_1 + E_2 + E_3\\
        \gamma mv &= E_1 -\frac{E_2}{2} -\frac{E_3}{2}\\
        0 &= \frac{\sqrt{3}E_2}{2} - \frac{\sqrt{3}E_3}{2}
    \end{align}
    From (3) we have that $E_3 = E_2$ hence by replacing in (1) and (2) we get
    that
    \begin{align}
        \gamma m &= E_1 + 2E_2\\
        \gamma mv &= E_1 - E_2
    \end{align}
    Then by subtracting the first equation from the last equation, we get that
    \begin{align*}
        \gamma mv -\gamma m &= E_1 - E_2 -(E_1 + 2E_2)\\
        \gamma m(v - 1) &= -3E_2\\
        E_2 &= \frac{\gamma m(1 - v)}{3}
    \end{align*}
    We know that $E_3 = E_2$ hence $E_3 = \frac{\gamma m(1 - v)}{3}$.
    Finally by replacing this value in (4) we get that
    \begin{align*}
        \gamma m &= E_1 + \frac{2\gamma m(1 - v)}{3}\\
        E_1 &= \gamma m\bigg(1 - \frac{2(1 - v)}{3}\bigg)
    \end{align*}

\end{proof}
\cleardoublepage
\begin{proof}{\textbf{3.32}}
    The 4-momentum vector before the collision (where we dropped the c's) is
    given by
    \begin{align*}
        P_p &= (E, E, 0, 0)\\
        P_M &= (M ,0 ,0 ,0)
    \end{align*}
    After the collision, the photon and the mass will have the following
    4-momentum vectors
    \begin{align*}
        P_{p}' &= (E', 0, E', 0)\\
        P_{M}' &= (\gamma M, p\cos\theta, -p\sin\theta, 0)
    \end{align*}
    Where we assumed the positive $y$-axis upwards in the direction of the
    photon after the collision.

    From the conservation of energy and momentum, we have that
    \begin{align*}
        E + M &= E' + \gamma M &\quad\text{(energy conservation)}\\
        E &= p \cos\theta &\quad\text{(x-axis momentum conservation)}\\
        0 &= E' - p\sin\theta &\quad\text{(y-axis momentum conservation)}
    \end{align*}
    From the x-axis and y-axis momentum conservation, we have that
    $E = p\cos\theta$ and $E' = p \sin\theta$ so the total momentum is given
    by $p^2 = (p\cos\theta)^2 + (p\sin\theta)^2 = E^2 + E'^2$.

    Also, we know that $(\gamma M)^2 = p^2 + M^2$ then by replacing
    $p^2$ we get that $(\gamma M)^2 = E^2 + E'^2 + M^2$.
    Finally, by replacing $\gamma M = E + M - E'$ that we got from the energy
    conervation equation we get that
    \begin{align*}
        (E + M - E')^2 &= E^2 + E'^2 + M^2\\
        (E+ M)^2 - 2(E+ M)E' + E'^2 &= E^2 + E'^2 + M^2\\
        E' &= -\frac{E^2 + M^2 - (E+M)^2}{2(E + M)}\\
        E' &= \frac{-E^2 - M^2 + E^2 +2EM + M^2}{2(E + M)}\\
        E' &= \frac{EM}{E+M}
    \end{align*}
\end{proof}
\cleardoublepage
\begin{proof}{\textbf{3.40}}
    Let us analyze the moment right before the collision then the acceleration
    of mass at this moment is $a = F/(\gamma^3 m)$ and we know that
    $a = v dv/dx$ hence the velocity at this point is
    \begin{align*}
        v\frac{dv}{dx} &= \frac{F}{\gamma^3 m}\\
        \int_{0}^v \gamma^3 vdv &= \frac{F}{m} \int_x^0 dx\\
        \frac{1}{\sqrt{1 - v^2}} - 1 &= -\frac{F}{m} x\\
        \frac{1}{\sqrt{1 - v^2}} &= \frac{m - Fx}{m}\\
        1 - v^2 &= \bigg(\frac{m}{m - Fx}\bigg)^2\\
        v &= \sqrt{1- \bigg(\frac{m}{m - Fx}\bigg)^2}
    \end{align*}
    Now let $M$ be the joint mass after the collision.
    So from the conservation of momentum we have that
    \begin{align*}
        p_M &= p_m + p_m = \gamma m v + 0 = \gamma m v
    \end{align*}
    And from the conservation of energy we get that
    \begin{align*}
        E_M = E_m + E_m = \gamma m + m = m(\gamma + 1)
    \end{align*}
    But also we know that $M^2 = E_M^2 - p_M^2$ hence
    \begin{align*}
        M^2 &= m^2(\gamma + 1)^2  - \gamma^2 m^2 v^2\\
        &= m^2(\gamma^2 + 2\gamma + 1  - \gamma^2 v^2)\\
        &= m^2(\gamma^2(1 - v^2) + 2\gamma + 1)\\
        &= 2m^2(1 + \gamma)
    \end{align*}
    Now by replacing the value we have for $v$ we get that
    \begin{align*}
        M^2 &= 2m^2\bigg(1 + \frac{1}{\sqrt{1 - (1-(\frac{m}{m - Fx})^2)}}\bigg)\\
        M^2 &= 2m^2\bigg(1 + \frac{1}{\frac{m}{m - Fx}}\bigg)\\        
        M^2 &= 2m^2\bigg(\frac{m + m- Fx}{m}\bigg)\\        
        M^2 &= 2m(2m- Fx)\\        
        M &= \sqrt{2m(2m - Fx)}
    \end{align*}
\end{proof}
\cleardoublepage
\begin{proof}{\textbf{3.45}}
    Let $l$ be the maximum distance the string will extend and let us assume
    the tension of the string $T$ is constant so we have that
    \begin{align*}
        F\Delta x &= \Delta E\\
        -T (l - 0) &= m - \gamma m\\
        l &= \frac{m(\gamma - 1)}{T} = \frac{m}{4T}
    \end{align*}
    Where we used that $\gamma_{3c/5} = 5/4$.
    Now we are interested in the distance $x$ at which the masses will meet
    again. At this point, the rear mass will have an energy of $m + Tx$,
    and the front mass will have an energy of $m + T(l - x)$.
    Also we know that $p = \sqrt{E^2 - m^2}$ so the momentum at the meeting
    point for each mass will be
    \begin{align*}
        p_{m_r} = \sqrt{(m + Tx)^2 - m^2}
        \quad\text{ and }\quad
        p_{m_f} = \sqrt{(5m/4 - Tx)^2 - m^2}
    \end{align*}
    But $F = dp/dt$ tells us that these magnitudes must be equal, because the
    same force $T$ (in magnitude, but opposite in direction) acts on the two
    masses for the same time hence
    \begin{align*}
        \sqrt{(m + Tx)^2 - m^2} &= \sqrt{\bigg(\frac{5m}{4} - Tx\bigg)^2 - m^2}\\
        m^2 + 2mTx + (Tx)^2 - m^2
        &= \bigg(\frac{5m}{4}\bigg)^2 - 2Tx\bigg(\frac{5m}{4}\bigg) + (Tx)^2 - m^2\\
        2mTx &= \bigg(\frac{5m}{4}\bigg)^2 - \frac{5mTx}{2}  - m^2\\
        2mTx + \frac{5mTx}{2} &= m^2\bigg(\frac{25}{16} - 1\bigg)\\
        Tx\bigg(\frac{9}{2}\bigg) &= m\bigg(\frac{9}{16}\bigg)\\
        x &= \frac{m}{8T}
    \end{align*}
    Therefore the masses will meet at a distance $x = m/8T$ from the starting
    point.
\end{proof}
\end{document}























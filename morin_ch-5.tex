\documentclass[11pt]{article}
\usepackage{amssymb}
\usepackage{amsthm}
\usepackage{enumitem}
\usepackage{amsmath}
\usepackage{bm}
\usepackage{adjustbox}
\usepackage{mathrsfs}
\usepackage{graphicx}
\usepackage{siunitx}
\usepackage[mathscr]{euscript}

\title{\textbf{Solved selected problems of Special Relativity - Morin}}
\author{Franco Zacco}
\date{}

\addtolength{\topmargin}{-3cm}
\addtolength{\textheight}{3cm}

\newcommand{\hatr}{\bm{\hat{r}}}
\newcommand{\hatx}{\bm{\hat{x}}}
\newcommand{\haty}{\bm{\hat{y}}}
\newcommand{\hatz}{\bm{\hat{z}}}
\newcommand{\hatth}{\bm{\hat{\theta}}}
\newcommand{\hatphi}{\bm{\hat{\phi}}}
\newcommand{\hatrho}{\bm{\hat{\rho}}}
\theoremstyle{definition}
\newtheorem*{solution*}{Solution}
\renewcommand*{\proofname}{Solution}

\begin{document}
\maketitle
\thispagestyle{empty}

\section*{Chapter 5 - General Relativity}

\begin{proof}{\textbf{5.1}}
    % Let $t_g$ be the time that passes on the clock that stays on the ground.
    Let us start by computing the time gain because of the General Relativity
    effect when the object is in motion (upward or downward) if we consider a
    ground-measured fraction of time $dt = dh/v$ we have that the time
    considering this effect is given by   
    \begin{align*}
        t_{gr} &= 2\int_0^h \Bigg(1 + \frac{gh}{c^2}\bigg)\frac{dh}{v}\\
        &= \frac{2}{v} \bigg(h + \frac{gh^2}{2c^2}\bigg) \\
        &= \frac{2h}{v} + \frac{gh^2}{vc^2}
    \end{align*}
    In addition, this is going to have a time loss due to the Special Relativity
    effects but since we are assuming that $v \ll c$ we can write $\gamma$ 
    by binomial approximation as $\gamma \approx 1 - (v^2/2c^2)$ hence
    \begin{align*}
        t_{sr} &= \gamma t_{gr}\\
        &= \bigg(1 - \frac{v^2}{2c^2}\bigg)
        \bigg(\frac{2h}{v} + \frac{gh^2}{vc^2}\bigg)\\
        &= \frac{2h}{v} + \frac{gh^2}{vc^2} - \frac{vh}{c^2} - \frac{vgh^2}{2c^4}
    \end{align*}
    Given that $v \ll c$ and the last term of $t_{sr}$ involves the 4th power
    of $c$ we can approximate $t_{sr}$ correctly with
    $$t_{sr}  = \frac{2h}{v} + \frac{gh^2}{vc^2} - \frac{vh}{c^2}$$
    Finally, we want to determine $T$ such that $t = (1 + gh/c^2)T + t_{sr}$
    where $t = 2h/v + T$ is the time that passed in the clock that stayed on 
    the ground, hence
    \begin{align*}
        \frac{2h}{v} + T &= \bigg(1 + \frac{gh}{c^2}\bigg)T
        + \frac{2h}{v} + \frac{gh^2}{vc^2} - \frac{vh}{c^2}\\
        0 &= \frac{gh}{c^2}T + \frac{gh^2}{vc^2} - \frac{vh}{c^2}\\
        ghT + \frac{gh^2}{v} &= vh\\
        T &= \frac{v}{g} - \frac{h}{v}
    \end{align*}

\end{proof}
\cleardoublepage
\begin{proof}{\textbf{5.2}}
\begin{itemize}
    \item [(a)] Let $f(h)$ be the time-dilation factor as a function of height
    then we know that $\Delta t_h = f(h) \Delta t_0$. Let us suppose that we
    have three clocks, one at the ground level, the next one at a height
    $h_1$ and the later one at a height $h_1 + h_2$ for some $h_2$.
    The time dilation between the one on the ground and the one at a height
    $h_1$ is $\Delta t_{h_1} = f(h_1) \Delta t_0$ and the time dilation
    between the one at a height $h_1$ and the one at a height
    $h_1 + h_2$ must be $\Delta t_{h_2} = f(h_2) \Delta t_{h_1}$ so by replacing
    $\Delta t_{h_1}$ we get that $\Delta t_{h_2} = f(h_2)f(h_1) \Delta t_{0}$
    but also the time dilation between the one on the ground and the one
    at a height $h_1 + h_2$ is
    $\Delta t_{h_2} = f(h_1 + h_2) \Delta t_{0}$ hence
    it must happen that $f(h_1 + h_2) = f(h_2)f(h_1)$.

    Let us derivate the equation $f(x + y) = f(x)f(y)$ with respect to $x$
    as follows 
    \begin{align*}
        f'(x + y) = f'(x)f(y)
    \end{align*}
    and let us set $x = 0$, then
    \begin{align*}
        f'(y) &= f'(0)f(y)
    \end{align*}
    but $f'(0)$ is a constant so let us call it $C$, then
    \begin{align*}
        \int \frac{f'(y)}{f(y)}dy &= C \int dy\\
        \log f(y) &= Cy + D\\
        f(y) &= e^{Cy}
    \end{align*}
    Where we used that $f(0) = 1$ to determine that $D = 0$.

    Let us take $C = g/c^2$ hence $f(h) = e^{gh/c^2}$ which agrees with the
    property we are looking for i.e.
    \begin{align*}
        f(h_1 + h_2) = e^{\frac{g(h_1 + h_2)}{c^2}}
        = e^{(gh_1/c^2)}e^{(gh_2/c^2)} = f(h_1)f(h_2)
    \end{align*}
    When $h$ is small we can approximate
    $e^{\frac{gh}{c^2}} \approx 1 + \frac{gh}{c^2}$ hence the time-dilation for a
    small $h$ would be
    \begin{align*}
        \Delta t_h = \bigg(1 + \frac{gh}{c^2}\bigg) \Delta t_0
    \end{align*}
    Which agrees with Eq. (5.4).
\cleardoublepage
    \item [(b)] Let us take a series of $n$ clocks 
    separated by a small height $\Delta h$ between the ground and a height $h$.
    Since the height between clocks is small we can use the approximated
    time-dilation factor, so for the first clock, we have that
    \begin{align*}
        \Delta t_{h_1} = \bigg(1 + \frac{g\Delta h}{c^2}\bigg) \Delta t_0
    \end{align*} 
    In the same way, the time dilation between the first clock and
    the second one is
    \begin{align*}
        \Delta t_{h_2} &= \bigg(1 + \frac{g\Delta h}{c^2}\bigg) \Delta t_{h_1}
            = \bigg(1 + \frac{g\Delta h}{c^2}\bigg)^2 \Delta t_0
    \end{align*}
    By continuing this process and replacing values subsequently, we can
    compute the time dilation for the $nth$ clock at height $h$ as follows
    \begin{align*}
        \Delta t_{h} = \bigg(1 + \frac{g\Delta h}{c^2}\bigg)^n \Delta t_{0}
    \end{align*}
    Hence the time dilation factor $f(h)$ for a height $h$ is 
    \begin{align*}
        f(h) = \bigg(1 + \frac{g\Delta h}{c^2}\bigg)^n
    \end{align*}
    By applying logarithms to both sides and using the approximation
    $\log(1 + x) \approx x$ we get that
    \begin{align*}
       \log(f(h))
       &= \log \bigg(\bigg(1 + \frac{g\Delta h}{c^2}\bigg)^n\bigg)\\
       \log(f(h)) &= n\log \bigg(1 + \frac{g\Delta h}{c^2}\bigg)\\
       \log(f(h)) &= n\bigg(\frac{g\Delta h}{c^2}\bigg)\\
       \log(f(h)) &= \frac{g h}{c^2}\\
       f(h) &= e^{\frac{g h}{c^2}}
    \end{align*}
    Which is the same factor we obtained in (a) as we wanted.
\end{itemize}
\end{proof}

\cleardoublepage
\begin{proof}{\textbf{5.3}}
    Let us take a series of $n$ clocks separated by a small distance $dr$
    between the radius $r_{low}$ and the radius $r_{high}$.
    Since the height between clocks is small we can use the approximated
    time-dilation factor, so for the first clock at a height
    $r_1 = r_{low} + dr$, we have that
    \begin{align*}
        \Delta t_{r_1} &=
        \bigg(1 + \frac{GMdr}{r_1^2c^2}\bigg)\Delta t_{r_{low}}
    \end{align*}
    We also used Newton's universal law of gravitation which gives the value of
    $g$ at radius $r$ as $g_r = GM/r^2$.

    In the same way, the time dilation between the first clock and
    the second one is
    \begin{align*}
        \Delta t_{r_2} &= \bigg(1 + \frac{GMdr}{r_2^2c^2}\bigg) \Delta t_{r_1}\\
        &= \bigg(1 + \frac{GMdr}{r_2^2c^2}\bigg)
            \bigg(1 + \frac{GMdr}{r_1^2c^2}\bigg) \Delta t_{r_{low}}
    \end{align*}
    Where $r_2 = r_1 + dr$.
    By continuing this process and replacing values subsequently, we can
    compute the time dilation for the $nth$ clock at $r_{high}$ as follows
    \begin{align*}
        \Delta t_{r_{high}} =
        \bigg(1 + \frac{GMdr}{r_1^2c^2}\bigg)
        \bigg(1 + \frac{GMdr}{r_2^2c^2}\bigg)
        ... \bigg(1 + \frac{GMdr}{r_n^2c^2}\bigg) \Delta t_{r_{low}}
    \end{align*}
    Hence the time dilation factor $f$ at $r_{high}$ is 
    \begin{align*}
        f =
        \bigg(1 + \frac{GMdr}{r_1^2c^2}\bigg)
        \bigg(1 + \frac{GMdr}{r_2^2c^2}\bigg)
        ... \bigg(1 + \frac{GMdr}{r_n^2c^2}\bigg)
    \end{align*}
    By applying logarithms to both sides we get that
    \begin{align*}
        \log(f)
            &= \log \bigg(
                \bigg(1 + \frac{GMdr}{r_1^2c^2}\bigg)
                \bigg(1 + \frac{GMdr}{r_2^2c^2}\bigg)
                ... \bigg(1 + \frac{GMdr}{r_n^2c^2}\bigg)
            \bigg)\\
        \log(f) &= 
            \log\bigg(1 + \frac{GMdr}{r_1^2c^2}\bigg) 
            + \log\bigg(1 + \frac{GMdr}{r_2^2c^2}\bigg)
            ... 
            + \log\bigg(1 + \frac{GMdr}{r_n^2c^2}\bigg)
    \end{align*}
    Using the approximation $\log(1 + x) \approx x$ and assuming $n \to \infty$ 
    we also approximated the sum to an integral hence
    \begin{align*}
        \log(f) &= \sum_{i=1}^n \frac{GMdr}{r_i^2c^2} = \frac{GM}{c^2}\sum_{i=1}^n \frac{dr}{r_i^2}\\
        \log(f) &= \frac{GM}{c^2} \int_{r_{low}}^{r_{high}} \frac{dr}{r^2}\\
        \log(f) &= \frac{GM}{c^2} \bigg[\frac{1}{r_{low}} - \frac{1}{r_{high}}\bigg]
    \end{align*}
    \cleardoublepage
    Finally, if we remove the logarithm and we approximate the exponential
    function using that $e^x \approx 1 +x$ we get that
    \begin{align*}
        f &= e^{\frac{GM}{c^2} \big[\frac{1}{r_{low}} - \frac{1}{r_{high}}\big]}\\
        f &= 1 + \frac{GM}{c^2} \bigg[\frac{1}{r_{low}} - \frac{1}{r_{high}}\bigg]
    \end{align*}
    Which is the factor we wanted.
\end{proof}
\begin{proof}{\textbf{5.4}}
    Because of problem 5.3, we know that the time elapsed on the satellite
    is more compared to the time elapsed on the ground due to the General
    Relativity effects. They are related by the following equation
    \begin{align*}
        \Delta t_{gr} =
        \bigg(1 + \frac{GM}{c^2} \bigg[\frac{1}{r_{low}} - \frac{1}{r_{high}}\bigg]\bigg)
        \Delta t_g
    \end{align*}
    We also need to take into account the Special Relativity effects, hence
    \begin{align*}
        \Delta t_{sr} &= \bigg(\sqrt{1 - \frac{v^2}{c^2}}\bigg)\Delta t_{gr}\\
        \Delta t_{sr} &=
        \bigg(\sqrt{1 - \frac{v^2}{c^2}}\bigg)
        \bigg(1 + \frac{GM}{c^2} \bigg[\frac{1}{r_{low}} - \frac{1}{r_{high}}\bigg]\bigg)
        \Delta t_g
    \end{align*}
    Where we used that $t_{sr} = t_{gr}/\gamma$ which is the satellite
    time-dilation (the proper time) we would get from an observer
    at the height of the satellite due to the Special Relativity effects.
    Therefore the factor by which a clock on a satellite runs faster
    relative to a clock on the surface of the earth is
    \begin{align*}
        f &=
        \bigg(\sqrt{1 - \frac{3900^2}{300,000,000^2}}\bigg)\\
        &\quad\times\bigg(1 +
        \frac{6.67 \times10^{-11} \cdot 6 \times 10^{24}}{(300,000,000)^2}
        \bigg[\frac{1}{6400\times10^3} - \frac{1}{26600\times10^3}\bigg]\bigg)\\
        f &= 0.9999999999154999 \times 1.0000000005276237\\
        f &= 1.0000000004431235
    \end{align*}
    Finally, if one day passed on the surface of the earth (86400 seconds)
    then in the satellite would have passed $86400.00003828587$ seconds
    which gives us a difference of $3.83 \times 10^{-5}$ seconds or
    $38.3 \mu s$.
\end{proof}
\cleardoublepage
\begin{proof}{\textbf{5.5}}
\begin{itemize}
    \item [(a)] In this case, $A$ just sees $B$ moving, so $A$ is not aware
    of any acceleration which implies that $A$ is only aware
    of the Special Relativity effect as follows
    \begin{align*}
        \Delta t_B &= \bigg(\sqrt{1 - \frac{v^2}{c^2}}\bigg)\Delta t_{A}
    \end{align*}
    Where we used that $t_B = t_{A}/\gamma$ which is $B$'s
    time-dilation (the proper time) we would get compared to
    $A$ due to the Special Relativity effects.
    Hence the the factor by which a clock on $B$ runs slower relative to a
    clock on $A$ is
    \begin{align*}
        f &= \sqrt{1 - \frac{v^2}{c^2}}\\
        f &= 1 - \frac{v^2}{2c^2}
    \end{align*}
    Where we used that $\sqrt{1 - v^2/c^2} \approx 1 - v^2/2c^2$ since we know
    that $v \ll c$.

    \item [(b)] From this reference frame, $B$ sees $A$ rotating with a
    velocity $v$ and $B$ being stationary experiences an acceleration away
    from $A$, i.e. like a gravitational effect, hence  
    the General Relativity effects must be taken into account.

    We know that the time elapsed on a stationary observer $A'$ at 
    a distance $r$ from $B$ is more compared to the
    time elapsed on $B$ due to the General Relativity effects
    and they are related by the following equation
    \begin{align*}
        \Delta t_{A'} =
        \bigg(1 + \frac{ar}{c^2}\bigg)
        \Delta t_B
    \end{align*}
    We also need to take into account the Special Relativity effects, hence
    \begin{align*}
        \Delta t_A &= \bigg(\sqrt{1 - \frac{v^2}{c^2}}\bigg)\Delta t_{A'}\\
        \Delta t_A &=
        \bigg(\sqrt{1 - \frac{v^2}{c^2}}\bigg)
        \bigg(1 + \frac{ar}{c^2}\bigg)
        \Delta t_{B}
    \end{align*}
    Where we used that $t_A = t_{A'}/\gamma$ which is $A$'s
    time-dilation (the proper time) we would get from a stationary observer
    $A'$ at a distance $r$ from $B$ due to the Special Relativity effects.
    \cleardoublepage
    Thus the factor by which a clock on $A$ runs faster relative to a
    clock on $B$ is
    \begin{align*}
        f &=
        \bigg(\sqrt{1 - \frac{v^2}{c^2}}\bigg)
        \bigg(1 + \frac{ar}{c^2}\bigg)\\
        f &=
        \bigg(1 - \frac{v^2}{2c^2}\bigg)
        \bigg(1 + \frac{v^2}{c^2}\bigg)\\
        f &= 1 + \frac{v^2}{c^2} - \frac{v^2}{2c^2}  - \frac{v^4}{2c^4}\\
        f &\approx 1 + \frac{v^2}{2c^2}
    \end{align*}
    Where we used that $\sqrt{1 - v^2/c^2} \approx 1 - v^2/2c^2$ and we
    removed the term which has the 4th power of $v$ since we know
    that $v \ll c$.
    But we are interested in the inverse factor which explains that a clock
    on $B$ runs slower relative to a clock on $A$, therefore we have that
    \begin{align*}
        f &= \frac{1}{1 + \frac{v^2}{2c^2}} \approx 1 - \frac{v^2}{2c^2}
    \end{align*}
    Where we used the binomial approximation again and the result matches
    to what we obtained in (a).
\cleardoublepage
    \item [(c)] In this case, $A$ sees $B$ stationary since
    its reference frame rotates with the same frequency as $B$, hence there
    is no Special Relativity effect to account for.
    
    Let us imagine then a series of $n$ clocks from $A$ to $B$ each
    separated by a distance $dx$.
    The gravitational acceleration for each clock where the first clock
    is at a distance $x_1$ from $A$ the second is at a distance $x_2$ and so on
    is $a = v_x^2/x_k = x_k\omega^2$ where we used that $v_x = x\omega$ also
    it's worth noticing that the acceleration is pointing outward from $A$
    hence we treat it as a negative acceleration.
    Hence for the first clock at a distance $dx$ from $A$ the time elapsed
    compared to a clock on $A$ will be
    \begin{align*}
        \Delta t_1 = \bigg(1 - \frac{x_1\omega^2dx}{c^2}\bigg)\Delta t_A
    \end{align*}
    In the same way for the $nth$ clock at $B$ we would have that
    \begin{align*}
        \Delta t_B &= \bigg(1 - \frac{x_n\omega^2dx}{c^2}\bigg)\Delta t_{n-1}\\
        \Delta t_B &=
        \bigg(1 - \frac{x_n\omega^2dx}{c^2}\bigg)...
        \bigg(1 - \frac{x_1\omega^2dx}{c^2}\bigg)
        \Delta t_{A}
    \end{align*}
    Thus the factor relating the time elapsed in $B$ with the time elapsed in
    $A$ is 
    \begin{align*}
        f &= \bigg(1 - \frac{x_n\omega^2dx}{c^2}\bigg)...
        \bigg(1 - \frac{x_1\omega^2dx}{c^2}\bigg)\\
        \log f &= \log \bigg(1 - \frac{x_n\omega^2dx}{c^2}\bigg) + ...+
        \log \bigg(1 - \frac{x_1\omega^2dx}{c^2}\bigg)\\
        \log f &= -\sum_{k = 1}^n \frac{x_k\omega^2dx}{c^2}\\
        \log f &= -\frac{\omega^2}{c^2} \int_0^r xdx
    \end{align*}
    Using the approximation $\log(1 + x) \approx x$ and assuming $n \to \infty$ 
    we also approximate the sum to an integral. Therefore, we have that
    \begin{align*}
        \log f &= -\frac{\omega^2}{c^2} \frac{r^2}{2}\\
        f &= e^{-\frac{v^2}{2c^2}}\\
        f &\approx 1 - \frac{v^2}{2c^2}
    \end{align*}
    Which also matches with the result we obtained in (a) and (b).
\end{itemize}
\end{proof}
\cleardoublepage
\begin{proof}{\textbf{5.12}}
\begin{itemize}
\item [(a)] Given that from the earth the periods of acceleration are
negligible the twin on the spaceship is younger compared to the twin on the
Earth by a factor of
\begin{align*}
    \sqrt{1 - \frac{v^2}{c^2}} \approx 1 - \frac{v^2}{2c^2} 
\end{align*}
Since from the Earth, the spaceship travels approximately a time $2l/v$
then the twin on the spaceship is
$$ \frac{2l}{v}\frac{v^2}{2c^2} = l \frac{v}{c^2}$$
much younger than the twin on Earth.

\item [(b)] From the spaceship frame, the twin on the spaceship sees the
twin on Earth traveling with a velocity $v$ hence if $t_s$ is the time we see
from the spaceship and $t_e$ is the proper time on Earth we have that 
\begin{align*}
    t_{s} &= \gamma t_e\\
    % t_e &= \bigg(1 - \frac{v^2}{2c^2}\bigg) t_s\\
    t_e &= \bigg(1 - \frac{v^2}{2c^2}\bigg) \frac{2l}{v}\\
    t_e &= \frac{2l}{v} - \frac{vl}{c^2}
\end{align*}
Where we used that $t_s = l/v$ for both directions of the journey.
This implies that the Earth time is less than the spaceship time by 
$vl/c^2$. But then the spaceship starts to deaccelerate for quite some time
hence because of the General Relativity effects we have that
\begin{align*}
    t_{e} &= \bigg(1 + \frac{al}{c^2}\bigg)t_s
\end{align*}
Also if we consider that $a = v/t'$ where $t' = t_s$ is the time we
deaccelerate according to the spaceship so we have that
\begin{align*}
    t_{e} &= t' + \frac{vl}{c^2}
\end{align*}
The same thing happens when the spaceship accelerates again hence the
earth time is above the spaceship time by $2vl/c^2$ so summing up the values
we get that the time on earth is above the time on the spaceship by
$2vl/c^2 - vl/c^2 = vl/c^2$. Therefore the twin on the spaceship is $vl/c^2$
much younger than the one on Earth which matches the result we got
earlier. 



\end{itemize}
\end{proof}
\cleardoublepage
\begin{proof}{\textbf{5.21}}
    We want to verify that $f = \gamma^3 ma$ using the equation we have for $v$.
    First, let us note that
    \begin{align*}
        v &= \frac{gt}{\sqrt{1 + (gt)^2}}\\
        1 - v^2 &= 1 - \frac{(gt)^2}{1 + (gt)^2}\\
        1 - v^2 &= \frac{1}{1 + (gt)^2}\\
        \frac{1}{\sqrt{1 - v^2}} &= \sqrt{1 + (gt)^2}
    \end{align*}
    This shows that $\gamma = \sqrt{1 + (gt)^2}$. Now we compute $a = dv/dt$
    as follows
    \begin{align*}
        a = \frac{dv}{dt} &= \frac{g}{(1 + (gt)^2)^{3/2}}
    \end{align*}
    Hence multiplying the equation by $m$ on both sides and using that
    $f = mg$ we get that
    \begin{align*}
        \frac{mg}{(1 + (gt)^2)^{3/2}} &= ma\\
        f &= (1 + (gt)^2)^{3/2}ma \\
        f &= \gamma^3ma
    \end{align*}
    Which is the equation we want.
\end{proof}
\cleardoublepage
\begin{proof}{\textbf{5.22}}
    Let us derivate first the velocity $v(\tau)$ i.e. the velocity of the
    particle at time $\tau$ (on the particle's frame) with respect to the
    original inertial frame.

    The velocity of the particle after some time $t + d\tau$
    with respect to the original inertial frame is given by the
    velocity addition formula. If the particle is moving at a velocity
    $gdt$ with respect to a previous inertial frame
    moving at a velocity $v(\tau)$ we see that
    \begin{align*}
        v(\tau + d\tau) = \frac{gd\tau + v(\tau)}{1 + v(\tau)gdt}
    \end{align*}
    The left-hand side can be written as $v(\tau) + dv$ so
    \begin{align*}
        v(\tau) + dv &= \frac{(gd\tau + v(\tau))}{(1 + v(\tau)gd\tau)}
        \frac{(1 - v(\tau)gd\tau)}{(1 - v(\tau)gd\tau)}\\
        v(\tau) + dv &= \frac{gd\tau - v(\tau)(gd\tau)^2 + v(\tau) - v(\tau)^2gd\tau}
        {1 - (v(\tau)gd\tau)^2}\\
        v + dv &= gd\tau + v - v^2gd\tau
    \end{align*}
    Where we removed the terms involving $d\tau^2$ and used $v(\tau) = v$ for
    simplicity. Hence we have that
    \begin{align*}
        dv &= (1 - v^2)gd\tau\\
        \int_0^v\frac{dv}{(1 - v^2)} &= g\int_0^\tau d\tau\\
        \tanh^{-1}(v) &= g\tau\\
        v &= \tanh(g\tau)
    \end{align*}
    Which is the equation we were looking for $v$.

    For $\gamma$ we know that
    \begin{align*}
        \gamma &= \frac{1}{\sqrt{1 - v^2}}
    \end{align*}
    Hence, replacing the value we have for $v$ we get that
    \begin{align*}
        \gamma &= \frac{1}{\sqrt{1 - \tanh^2(g\tau)}} = \cosh(g\tau)
    \end{align*}

    Finally, for small periods, the time in the original inertial frame
    is related to the time measured on the particle by the time-dilation
    equation hence 
    \begin{align*}
        dt &= \gamma d\tau
    \end{align*}
    Therefore we have that
    \begin{align*}
        \int_0^t dt &= \int_0^\tau\cosh(g\tau) d\tau\\
        gt &= \sinh(g\tau)
    \end{align*}
    Which is the last equation we wanted.
\end{proof}
\cleardoublepage
\begin{proof}{\textbf{5.28}}
    Let us consider a function
    $y(t) = y_0(t) + \varepsilon(t)$ where $y_0$
    is the path that yields the stationary value and $\varepsilon$ is a small
    variation. Let us name the action integral as
    $S(y) = \int_{t_1}^{t_2} \big(\frac{mv^2}{2} - mgy\big)dt$. We want
    to show that $S(y_0)$ is a global minimum for $S$ so let us compute first
    $S(y_0 + \epsilon)$ as follows
    \begin{align*}
        S(y_0 + \varepsilon) 
        &= \int_{t_1}^{t_2} \bigg(\frac{m(\dot{y_0}(t) + \dot{\varepsilon}(t))^2}{2}
        - mg(y_0(t) + \varepsilon(t))\bigg)dt\\
        &= \frac{m}{2}\int_{t_1}^{t_2}
        \bigg(\dot{y_0}^2 + 2\dot{y_0}\dot{\varepsilon} +\dot{\varepsilon}^2
        - 2g(y_0 + \varepsilon)\bigg)dt\\
        &= \frac{m}{2}\int_{t_1}^{t_2}
        \bigg(\dot{y_0}^2 +\dot{\varepsilon}^2 - 2g(y_0 + \varepsilon)\bigg)dt
        + m\int_{t_1}^{t_2} \dot{y_0}\dot{\varepsilon} dt
    \end{align*}
    Integrating by parts the last term and using that $\varepsilon$ is $0$ 
    at the boundaries, we get that
    \begin{align*}
        S(y_0 + \epsilon)
        &= \frac{m}{2}\int_{t_1}^{t_2}
        \bigg(\dot{y_0}^2 +\dot{\varepsilon}^2 - 2g(y_0 + \varepsilon)\bigg)dt
        + m\bigg(\bigg[\dot{y_0}\varepsilon\bigg]_{t_1}^{t_2} -
        \int_{t_1}^{t_2} \ddot{y_0}\varepsilon dt\bigg)\\
        &= \frac{m}{2}\int_{t_1}^{t_2}
        \bigg(\dot{y_0}^2 +\dot{\varepsilon}^2 - 2g(y_0 + \varepsilon)
        - 2\ddot{y_0}\varepsilon\bigg)dt\\
        &= \frac{m}{2}\int_{t_1}^{t_2}
        \bigg(\dot{y_0}^2 +\dot{\varepsilon}^2 - 2gy_0 - 2\varepsilon(
        g + \ddot{y_0})\bigg)dt
    \end{align*}
    Since we want to show that $S(y_0)$ is the global minimum let us compute
    $S(y_0 + \varepsilon) - S(y_0)$ as follows
    \begin{align*}
        S(y_0 + \epsilon) - S(y_0)
        &= \frac{m}{2}\int_{t_1}^{t_2}
        \bigg(\dot{y_0}^2 +\dot{\varepsilon}^2 - 2gy_0 - 2\varepsilon(
        g + \ddot{y_0})\bigg)dt\\
        &\quad\quad- \frac{m}{2}\int_{t_1}^{t_2}
        \bigg(\dot{y_0}^2 - 2gy_0\bigg)dt\\
        &= \frac{m}{2}\int_{t_1}^{t_2}
        \bigg(\dot{\varepsilon}^2 - 2\varepsilon(g + \ddot{y_0})\bigg)dt
    \end{align*}
    On the other hand, we know $y_0$ makes $S$ stationary then by definition
    we have that $S(y_0 + \varepsilon) - S(y_0) = O(\varepsilon^2)$ which
    implies that the integral term that depends on $\varepsilon$ must be $0$
    hence
    \begin{align*}
        S(y_0 + \epsilon) - S(y_0)
        &= \frac{m}{2}\int_{t_1}^{t_2} \dot{\varepsilon}^2dt
    \end{align*}
    Finally, since the integral of a positive function must be positive
    we have that $S(y_0 + \epsilon) - S(y_0) \geq 0$ which implies that
    $S(y_0)$ is a global minimum for $S$ as we wanted.
\end{proof}

THE END.
\end{document}


